\chapter{Anneaux}

\section{Rappels sur les anneaux commutatifs}

\begin{definition}
	Un anneau commutatif est un groupe abélien $(A, +)$ muni d'une application:

	\begin{equation*}
		A \cartprod A \rightarrow A : (a, b) \rightarrow a.b = ab
	\end{equation*}

	tel que:

	\begin{itemize}
		\item $\forall a, b, c \in A$, $(ab)c = a(bc)$.
		\item $\exists 1_{A} \in A$, $\forall a \in A$, $1_{A} a = a 1_{A} = a$.
		\item $\forall a, b \in A$, $ab = ba$ (commutativité).
		\item $\forall a, b, c \in A$, $a(b + c) = ab + ac$. (distributivité).
	\end{itemize}
\end{definition}

\begin{exemple}
	$\integer$, $\real$, $\rational$, $\complex$, $\integer[X]$, $\real[X]$,
	$\mathbb{F}_{p} = \integer/p\integer$.
\end{exemple}

\begin{proposition}
	Soit $A$ un anneau commutatif. Alors $A[X]$ est un anneau commutatif.
\end{proposition}

\ifdefined\outputproof
\begin{proof}

\end{proof}
\fi

\begin{remarque}
	$(A[X])[Y] = A[X, Y]$, ie les polynomes à deux variables sont les polynomes
	construit sur un anneau de polynomes.
\end{remarque}

\begin{proposition}
	Soient $A$ et $B$ deux anneaux commutatifs. Alors $A \cartprod B$ est un
	anneau commutatif (avec la multiplication composante par composante).
\end{proposition}

\ifdefined\outputproof
\begin{proof}

\end{proof}
\fi

\begin{proposition}
	Soit $\GSsequence{A}{i}{I}$ une famille d'anneaux commutatifs. Alors
	$\displaystyle \prod_{i \in I} A_{i}$ est un anneau commutatif.
\end{proposition}

\ifdefined\outputproof
\begin{proof}

\end{proof}
\fi

\begin{exemple}
	$\rational^{\naturel}$ est un anneau commutatif.
\end{exemple}

\begin{remarque}
	\begin{itemize}
		\item $1_{A}$ est unique.
		\item $\forall a \in A$, $0_{A} a = 0_{A} = a 0_{A}$.
		\item $-1_{A} a = -a$.
		\item $0_{A} = 1_{A} \equiv A = \GSset{0_{A}}$.
		\item Si on enlève la distributivité à gauche dans la définition, alors
			on obtient la définition d'un anneau. Exemple : $M_{n}(\real)$,
			$M_{n}(\complex)$, $\GSendomorphism{G}$ ($G$ groupe abélien).
	\end{itemize}
\end{remarque}

\begin{definition}
	On définit $A^{\cartprod}$ comme l'ensemble des inversibles de $A$, c'est-à-dire
	que:

	\begin{equation*}
		A^{\cartprod} = \GSsetDef{a \in A}{\exists b \in A, ab = ba = 1_{A}}
	\end{equation*}
\end{definition}

\begin{remarque}
	Si $(A, .)$ est un monoïde, alors $(A^{\cartprod}, .)$ est un groupe
	abélien.
\end{remarque}

\begin{exemple}
	$\integer^{\cartprod} = \GSset{1, -1}$, $\rational^{\cartprod} = \rational
	\textbackslash \GSset{0}$, $\real^{\cartprod} = \real \textbackslash
	\GSset{0}$, $\real[X]^{\cartprod} = \real^{\cartprod}$,
	$\integer[X]^{\cartprod} = \integer^{\cartprod}$.
\end{exemple}

\begin{definition}
	\textbf{Un corps} est un anneau non nul tel que $A^{\cartprod} = A
	\textbackslash \GSset{0_{A}}$.
\end{definition}

\begin{definition}
	\textbf{Un sous-anneau} $B$ de $A$ est un sous-groupe $(B, +)$ tel que
	$1_{A} \in B$ et $\forall a, b \in B$, $ab \in B$.
\end{definition}


\begin{proposition}
	Soient $B$ et $C$ deux sous-anneaux de $A$, alors $B \inter C$ est un
	sous-anneau de $A$. La propriété reste vraie pour une intersectionq
	quelconque.
\end{proposition}

\ifdefined\outputproof
\begin{proof}

\end{proof}
\fi

\begin{proposition}
	Soit $B$ sous-anneau de $A$, alors $B^{\cartprod}$ sous-anneau de
	$A^{\cartprod}$.
\end{proposition}

\ifdefined\outputproof
\begin{proof}

\end{proof}
\fi

\begin{definition}
	$\integer[i] = \GSsetDef{a + bi}{a, b \in \integer}$ est l'ensemble des
	entiers de Gauss.
\end{definition}

\begin{proposition}
	$\integer[i]$ est un sous-anneau de $\complex$, dont les élements
	inversibles sont $1, -1, i, -i$.
\end{proposition}

\ifdefined\outputproof
\begin{proof}

\end{proof}
\fi

\begin{proposition}
	$\rational[i] = \GSsetDef{a + bi}{a, b \in \rational}$ est un sous-corps de
	$\complex$ contenant $\rational$.
\end{proposition}

\ifdefined\outputproof
\begin{proof}

\end{proof}
\fi

\begin{definition}
	\textbf{Un morphisme d'anneau} $f$ entre deux anneaux $A$ et $B$ est un
	morphisme de groupe tel que $f(ab) = f(a) f(b)$ et $f(1_{A}) = 1_{B}$.

	De plus, c'est
	\begin{itemize}
		\item \textbf{un endormorphisme} si $A = B$.
		\item \textbf{un isomorphisme} si $f$ est bijectif.
		\item \textbf{un automorphisme} si $f$ est un isomorphisme et un
			endormorphisme.
		\item \textbf{un morphisme de corps} si c'est un morphisme d'anneau entre deux corps.
	\end{itemize}
\end{definition}

\begin{remarque}
	Si $\GSfunction{f}{A}{B}$ est un morphisme d'anneau, alors $f(A^{\cartprod})
	\subseteq B^{\cartprod}$, et
	\begin{equation}
		f^{*} : (A^{\cartprod}, .) \rightarrow (B^{\cartprod}, .)
	\end{equation}
	est un morphisme de groupe.
\end{remarque}

\begin{proposition}
	Il existe un unique morphisme d'anneau entre $\rational$ (resp $\integer$)
	et $\complex$.
\end{proposition}

\ifdefined\outputproof
\begin{proof}
	Si on a un morphisme f, on a par récurrence $\forall n \in \integer, f(n) =
	n$. D'où, le seule morphisme est $({Id_{\complex}})_{|\integer}$.

	La démonstration est la même pour $\rational$.
\end{proof}
\fi

\begin{proposition}
	Il existe seulement deux morphismes d'anneaux entre $\rational[i]$ (resp
	$\integer[i]$) et $\complex$.
\end{proposition}

\ifdefined\outputproof
\begin{proof}

\end{proof}
\fi

\begin{corollary}
	\begin{enumerate}
		\item $\GSautomorphismDef{corps}{\rational} =
			\GSautomorphismDef{anneau}{\rational} = \GSset{Id}$
		\item $\GSautomorphismDef{anneau}{\integer} = \GSset{Id}$
	\end{enumerate}
\end{corollary}

\ifdefined\outputproof
\begin{proof}
	En effet, si d'autres automorphismes existeraient, on pourrait étendre
	l'ensemble d'arrivé en $\complex$, et ils resteraient des morphismes.
\end{proof}
\fi

\begin{question}
	Combien il y a d'automorphismes de corps sur $\complex$ ? Sur $\real$ ?
	\href{http://www.math.uga.edu/~pete/Kestelman51.pdf}{Ici pour $\complex$}

	Pour $\real$, le seul automorphisme continu est l'identité. En effet, comme
	$\rational$ est dense dans $\real$, on a une suite d'élément de $\rational$
	qui tend vers $x \in \real$. On a alors que $f(x)$ est la limite de la suite
	$f(x_{n})$, où f maintenant est un morphisme de $\rational$ dans $\real$.
	Or, le seul morphisme est l'identité, donc $f(x) = x$ à la limite.

	Pour $\complex$, les seuls automorphismes sont l'identité et la conjugaison
	(même raisonnement que pour $\real$ en décomposant l'image de f en partie
	réelle et partie imaginaire).
\end{question}

\begin{proposition}
	Soient $A, B, C$ trois anneaux.
	Soient $\GSfunction{f}{A}{B}$ et $\GSfunction{g}{B}{C}$ deux morphismes
	d'anneaux. Alors $g \circ f : A \rightarrow C$ est un morphisme d'anneau.
\end{proposition}

\ifdefined\outputproof
\begin{proof}

\end{proof}
\fi

\begin{proposition}
	\begin{itemize}
		\item $\GSfunction{f}{A}{B}$ isomorphisme d'anneau $\implies
			\GSfunction{f^{-1}}{B}{A}$ isomorphisme d'anneau.
		\item $\GSfunction{Id_{A}}{A}{A} : a \rightarrow a$ est un isomorphisme
			d'anneau.
		\item $(\GSautomorphismDef{anneau}{A}, \circ)$ est un groupe.
		\item Si $A$ est un corps, $\GSautomorphismDef{anneau}{A} =
			\GSautomorphismDef{corps}{A}$.
		\item Si $\GSfunction{f}{A}{B}$ est un morphisme d'anneau, alors
			$Im(f)$ est un sous-anneau de $B$ et $\ker(f)$ est un sous-groupe de
			$A$.
	\end{itemize}
\end{proposition}

\ifdefined\outputproof
\begin{proof}

\end{proof}
\fi

\begin{definition}
	Soit $A$ un anneau. \textbf{Un idéal de $A$} est un sous-groupe $(I, +)$ de
	$(A, +)$ tel que $\forall a \in I$, $\forall b \in A$, $ab \in I$.
\end{definition}

\begin{exemple}
	-- $\GSset{0_{A}}$ et $A$ sont des idéaux de $A$.
\end{exemple}

\begin{proposition}
	Soit $\GSfunction{f}{A}{B}$ où $A$ et $B$ sont deux anneaux commutatifs. Alors
	Im(f) est un sous-anneau de B, et ker(f) est un idéal de A.

	De plus on a :
	\begin{enumerate}
		\item Si $Im(f)$ est un idéal, $f$ est surjectif.
		\item Si $Ker(f)$ est un sous-anneau, $B = \GSset{0_{B}}$.
	\end{enumerate}
\end{proposition}

\ifdefined\outputproof
\begin{proof}
	$1$. On doit montrer que$ Im(f) = B$. Comme $f$ morphisme, $1 \in Im(f)$, on a
	donc que $Im(f) = B$ car $Im(f)$ est un idéal.
	$2$.
\end{proof}
\fi

\begin{proposition}
	Les idéaux de $\integer$ sont les $\integer/n\integer$. Les idéaux de
	$\integer$ sont donc confondus avec les sous-groupes de $(\integer, +)$.
\end{proposition}

\ifdefined\outputproof
\begin{proof}

\end{proof}
\fi

\begin{proposition}
	Soit I et J deux idéaux d'un anneau commutatifs A tel que $I \subseteq J$.
	Alors J/I est un idéal de A/I.
\end{proposition}

\ifdefined\outputproof
\begin{proof}

\end{proof}
\fi

\begin{definition} [Idéal principal]
	\label{def:principal_ideal}
	Un idéal $I$ est un \textbf{idéal principal} si il est engendré par un seul
	élément. On note, si $a \in A$ engendre $I$, $I = (a)$.
\end{definition}

\begin{definition} [Anneau principal]
	Un anneau $A$ est un \textbf{anneau principal} si tout idéal est principal.
	\label{principal_ring}
\end{definition}

\begin{exemple}
	\begin{itemize}
		\item $\GSset{0_{A}} = (0_{A})$, $A = (1_{A})$
		\item $(2)_{\integer} = 2\integer \neq \integer$.
		\item $(2)_{\rational} = 2\rational = \rational$.
		\item $I = \GSsetDef{2 P(X) + X Q(X)}{P, Q \in \integer[X]}$ est un
			idéal de $\integer[X]$ non principal.
	\end{itemize}
\end{exemple}

\begin{definition}
	Soit $A$ un anneau. Soient $a, b \in A$. On dit que \textbf{$a$ divise $b$}
	s'il existe $n \in \integer$ tel que $b = na$.
\end{definition}

\begin{proposition}
	$a$ divise $b$ $\equiv (b) \subseteq (a)$.
\end{proposition}

\ifdefined\outputproof
\begin{proof}

\end{proof}
\fi

\begin{exemple}
	Soit $n, m \in \integer$. Alors $n | m \equiv \integer/m\integer \subseteq
	\integer/n\integer$.
\end{exemple}

\begin{proposition}
	Soient $I, J$ deux idéaux d'un anneau $A$. alors

	\begin{itemize}
		\item $I \inter J$ est le plus grand idéal de $A$ contenu dans $I$ et $J$
		\item $I + J$ est le plus petit idéal contenant $I$ et $J$ (en
			particulier, c'est le plus petit sous-groupe contenant $I$ et $J$).
	\end{itemize}
	Nous pouvons généraliser à un nombre quelconque d'idéaux.
\end{proposition}

\ifdefined\outputproof
\begin{proof}

\end{proof}
\fi

\begin{proposition}
	Si $n, m \in \integer \textbackslash \GSset{0}$, alors $n\integer \inter
	m\integer = ppcm(n, m)\integer$ et $n\integer + m\integer = pgcd(n, m)$
\end{proposition}

\ifdefined\outputproof
\begin{proof}

\end{proof}
\fi

\begin{proposition}
	Soit $I$ un idéal de $A$. Alors, $I = A \equiv I \inter A^{\cartprod} \neq
	\emptyset \equiv 1_{A} \in I$.
\end{proposition}

\ifdefined\outputproof
\begin{proof}

\end{proof}
\fi

\begin{proposition}
	Les seuls idéaux d'un corps $\mathbb{K}$ sont $(0_{\mathbb{K}})$ où
	$\mathbb{K}$.
\end{proposition}

\ifdefined\outputproof
\begin{proof}
	Soit $I$ un idéal. S'il est nul, on a fini. Sinon on a un élément non
	nul $a \in I$. Comme $K$ est un corps, $a^{-1} \in K$, et donc $a^{-1}a = 1
	\in I$ car $I$ idéal. Donc $I = K$.
\end{proof}
\fi

\begin{corollary}
	Soit $\mathbb{K}$ un corps et L un anneau, et
	$\GSfunction{f}{\mathbb{K}}{L}$ un morphisme d'anneau. Alors soit L est nul,
	soit f est injectif.
\end{corollary}

\ifdefined\outputproof
\begin{proof}

\end{proof}
\fi

\begin{exemple}
	Soient $\mathbb{K}$ un corps et $E$ un espace vectoriel non-nul sur
	$\mathbb{K}$.

	Alors l'application
	\begin{equation}
		f : \mathbb{K} \rightarrow End_{\mathbb{K}}(E) : \lambda \rightarrow
		f(\lambda)
	\end{equation}
	où
	\begin{equation}
		f(\lambda) : E \rightarrow E : v \rightarrow \lambda v
	\end{equation}
	est un morphisme d'anneau injectif.
\end{exemple}

\section{Quotients}

\begin{proposition}
	Soit $A$ un anneau commutatif et soit $I$ un idéal de $A$.
	La multiplication induit un sur $A$ induit une structure d'anneau sur $(A/I,
	+)$ où
	\begin{equation}
		(a + I) (b + I) = ab + I
	\end{equation}
	et la projection
	\begin{equation}
		\pi_{I} : A \rightarrow A / I : a \rightarrow a + I
	\end{equation}
	est un morphisme d'anneau surjectif.

	De plus,
	\begin{itemize}
		\item $\ker(\pi_{I}) = I$
		\item $(A / I)^{\cartprod} = \GSsetDef{a + I \in A / I}{\exists b \in A
				\text{ tel que } ab^{-1} = I}$
	\end{itemize}
\end{proposition}

\ifdefined\outputproof
\begin{proof}

\end{proof}
\fi

\begin{exemple}
	\begin{itemize}
		\item $n \in \integer_{0}$, $A = \integer$, $I = n \integer \rightarrow
			\integer / n \integer$ anneau et
			\begin{align}
				(\integer / n \integer)^{\cartprod} &= \GSsetDef{a +
				n\integer}{\exists b \in \integer, ab^{-1} \in n\integer} \\
				& = \GSsetDef{a + n \integer}{\exists b, c \in \integer \text{
				tel que } a b + n c = 1_{A}} \\
				& = \GSsetDef{a + n \integer}{pgcd(a, n) = 1}
			\end{align}
		\item $(\integer / 6 \integer)^{\cartprod} = \GSset{1 + 6\integer = 6\integer, -1 +
			6\integer = 5 + 6 \integer}$
	\end{itemize}
\end{exemple}

\begin{proposition}
	Soient $A$ et $B$ deux anneaux commutatifs et soit $\GSfunction{f}{A}{B}$ un
	morphisme d'anneau.
	Soit $I$ un idéal de $A$.

	Alors il existe un morphisme d'anneau
	\begin{equation}
		\GSfunction{\overline{f}}{A / I}{B}
	\end{equation}
	vérifiant
	\begin{equation}
		\overline{f} \circ \pi_{I} = f \equiv I \subseteq \ker(f)
	\end{equation}

	Dans ce cas, on a
	\begin{itemize}
		\item $\Im(\overline{f}) = Im(f)$
		\item $\ker(\overline{f}) = \ker(f) / I$
	\end{itemize}
\end{proposition}

\ifdefined\outputproof
\begin{proof}

\end{proof}
\fi

\begin{corollary}
	Soient $A$, $B$ deux anneaux et soit $\GSfunction{f}{A}{B}$ un morphisme
	d'anneaux.

	Alors
	\begin{equation}
		\overline{f} : A / \ker(f) \rightarrow \Im(f) : a + \ker(f) \rightarrow
		f(a)
	\end{equation}
	est un isomorphisme d'anneaux
\end{corollary}

\ifdefined\outputproof
\begin{proof}

\end{proof}
\fi

\section{Morphismes fondamentaux}

\subsection{Caractéristique d'un anneau}
\label{subsection:ring_caracteristic}

\begin{proposition}
	Soit $A$ un anneau commutatif. Il existe un unique morphisme d'anneau
	$\GSfunction{\mu_{A}}{\integer}{A} : n \integer n 1_{A}$ et $\exists c_{A} \in
	\naturel$ tel que $\ker{\mu_{A}} = c_{A} \integer$.
\end{proposition}

\ifdefined\outputproof
\begin{proof}

\end{proof}
\fi

\begin{definition}
	$c_{A}$ est appelé \textbf{la caractéristique de $A$}.
\end{definition}

\begin{exemple}
	La caractéristique de $\integer/4\integer \cartprod \integer/6\integer$ vaut
	$12$.
\end{exemple}

\subsection{Anneaux de polynomes}

\begin{proposition}
	Soit $\GSfunction{f}{A}{B}$ un morphisme d'anneaux. Alors

	\begin{equation}
		\GSfunction{\tilde{f}}{A[X]}{B[X]} : P(X) = \sum_{i = 1}^{k} a_{k} X^{k}
	\rightarrow \sum_{i = 1}^{k} f(a_{k}) X^{k}
	\end{equation}
	est un morphisme d'anneaux. De plus, $\tilde{f}_{|A} = f$.
\end{proposition}

\ifdefined\outputproof
\begin{proof}

\end{proof}
\fi

\begin{exemple}
	\begin{itemize}
		\item La conjugaison complexe étant un automorphisme d'anneaux de
			$\complex$, la fonction
			\begin{equation}
				\overline{f} : \complex[X] \rightarrow \complex[X] : \sum_{i =
				1}^{n} a_{i} X^{i} \rightarrow \conjuguate{a_{i}} X^{i}
			\end{equation}
			est un automorphisme d'anneaux.
		\item Soit $n \in \integer$ et soit $\pi_{n} : \integer \rightarrow
			\integer / n \integer : a \rightarrow a + mod(n \integer)$ le morphisme surjectif de
		projection.

		Alors
		\begin{align}
			\tilde{\pi_{n}} : \integer[X] & \rightarrow \integer / n \integer[X]
			\\
			\sum_{i = 1}^{n} a_{i} X^{i} & \rightarrow \sum_{i = 1}^{n} (a_{i}
			mod(n \integer)) X^{i}
		\end{align}
		est un morphisme d'anneau surjectif.

		Par factorisation, on obtient l'isomorphisme d'anneau
		\begin{equation}
			\integer[X] / n \integer[X] \xrightarrow{\sim} (\integer
			/ n \integer)[X]
		\end{equation}

	\end{itemize}
\end{exemple}

\subsection{Le corps des réels}

\begin{proposition}
	Soit $\mathcal{C}(\rational) =
	\GSsetDef{\GSsequence{a}{n}{\naturel}}{a_{n} \text{ est une suite de Cauchy}}$.
	Alors $\mathcal{C}(\rational)$ est un sous-anneau de $\rational^{\naturel}$
\end{proposition}

\ifdefined\outputproof
\begin{proof}

\end{proof}
\fi

\begin{definition}
	$\mathcal{C}_{0}(\rational) = \GSsetDef{\GSsequence{a}{n}{\naturel}}{lim
		a_{n} = 0}$
\end{definition}

\begin{proposition}
	$\mathcal{C}_{0}(\rational)$ est un idéal de $\rational^{\naturel}$.
\end{proposition}

\ifdefined\outputproof
\begin{proof}

\end{proof}
\fi

\begin{definition}
	$\real = \rational^{\naturel} / \mathcal{C}_{0}(\rational)$
\end{definition}

\subsection{Evaluation interne}

\begin{proposition}
	Soit $A$ un anneau commutatif. Soit $a \in A$.

	Alors 
	\begin{equation}
		\GSfunction{eval_{a}}{A[X]}{A} : P(X) \rightarrow P(a)
		\label{<++>}
	\end{equation}
	est un morphisme d'anneau surjectif et ${(eval_{a})}_{|A} = Id_{A}$.
\end{proposition}

\ifdefined\outputproof
\begin{proof}

\end{proof}
\fi

\begin{proposition}
	Pour tout $a \in A$, 
	\begin{equation}
		\GSfunction{\tau_{a}}{A[X]}{A[X]} : P(X) \rightarrow P(X - a)
	\end{equation}
	est un automorphisme d'anneau avec $(\tau_{a})^{-1} = \tau_{-a}$.
\end{proposition}

\ifdefined\outputproof
\begin{proof}

\end{proof}
\fi

\begin{proposition}
	Pour tout $a \in A$, 

	\begin{equation}
		eval_{a} \circ \tau_{a} = eval_{0}
	\end{equation}

\end{proposition}

\ifdefined\outputproof
\begin{proof}

\end{proof}
\fi

\begin{corollary}
	$\ker(eval_{a}) = \tau_{a}(\ker(eval_{0})) = (X - a)$
\end{corollary}

\ifdefined\outputproof
\begin{proof}

\end{proof}
\fi

\begin{proposition}
	Pour tout $a \in A$, l'application
	\begin{equation}
		A[X] / (X - a) \rightarrow A : P(X) \, mod(X - a) \rightarrow P(a)
	\end{equation}
	est un isomorphisme d'anneaux.

	On a donc $A[X] / (X - a) \xrightarrow{\sim} A$.
\end{proposition}

\ifdefined\outputproof
\begin{proof}

\end{proof}
\fi

\subsection{Evaluation externe}

\begin{proposition}
	Soient $A$ et $B$ deux anneaux commutatifs tel que $A$ est un sous-anneau de
	$B$. Alors, pour tout $b \in B$, l'application d'évaluation en $b$
	\begin{equation}
		eval_{b} : A[X] \rightarrow B : P(X) \rightarrow P(b)
	\end{equation}
	est un morphisme d'anneau tel que ${(eval_{b})}_{|A}$ est
	l'inclusion de $A$ dans $B$.
	
	De plus, $Im(eval_{b}) := A[b] := $ le plus petit sous-anneau de $B$
	contenant $A$ et $b$. % TODO noyau ? pas dans le resumé de Sam.
\end{proposition}

\ifdefined\outputproof
\begin{proof}

\end{proof}
\fi

% TODO exemple page 8

\section{Théorème chinois}

%\subsection{Aspect historique}
% Aspect historique

\subsection{Énoncé}
% Théorème + preuve
Nous souhaitons, pour deux anneaux donnés, et sous certaines conditions, montrer
que $A/(I \cap J)$ est isomorphe à $A/I \times A/J$.

Soient $I, J$ deux idéaux de $A$. Alors, nous pouvons construire

\begin{equation}
	\GSfunction{f}{A}{A/I \times A/J} : a \rightarrow (a + I, a + J)
\end{equation}

Nous avons $ker(f) = I \cap J$.

On a alors un morphisme injectif 
\begin{equation}
	\GSfunction{\overline{f}}{A/(I \cap J)}{A/I \times A/J}
\end{equation}
induit par $f$.

Il nous manque donc une condition, la surjectivité, pour avoir un isomorphisme
entre ces deux anneaux.
On en vient alors au théorème chinois :

\begin{theorem}
	Si $I + J = A$, alors $\overline{f}$ est surjectif, et donc $A/(I \cap
	J)$ est isomorphe à $A/I \times A/J$.
\end{theorem}

\ifdefined\outputproof
\begin{proof}

\end{proof}
\fi

On peut généraliser ce théorème à un nombre fini d'idéaux $I_{1}, \ldots,
I_{n}$.

On construit comme précédemment $f$, et on déduit un morphisme injectif
$\overline{f}$. Il suffit de trouver une condition pour que $\overline{f}$. Nous
avons alors le résultat suivant, dit théorème chinois généralisé.

\begin{theorem}
	Si pour tout $1 \leq k \leq n$, on a
	
	\begin{equation}
		I_{k} + \displaystyle \bigcap_{{i = 1, i \neq k}}^{n} I_{i} = A
	\end{equation}

	alors $\overline{f}$ est surjectif, et
	donc $\GSprodSet{i}{1}{n}{A/I}$ est isomorphe à $\displaystyle A/\bigcap_{i = 1}^{n}
	I_{i}$.
\end{theorem}

\ifdefined\outputproof
\begin{proof}

\end{proof}
\fi

\subsection{Interprétation et applications}

\begin{exemple}
	\begin{itemize}
		\item Soient $A = \integer$, $I = n \integer$, $J = m\integer$. On a $I
			\inter J = ppcm(n, m) \integer$, et $I + J = pgcd(n, m) \integer$.
			D'où $I + J = \integer \equiv pgcd(n, m) = 1 \equiv ppcm(n, m) =
			nm$.

			Donc, si $pgcd(n, m) = 1$, on a un isomorphisme entre $\integer / nm
			\integer$ et $\integer / n \integer \cartprod \integer / m
			\integer$.

			Celui-ci est donné par $a \, \, mod(nm \integer) \rightarrow (a \,
			\, mod(n
			\integer), a \, \, mod(m \integer))$.
		\item Soient $K$ un anneau commutatif non nul, $A = K[X]$, $I = (X)$, $J
			= (X - 1)$. On a $I + J = K[X]$ et $I \inter J = (X^{2} - X)$.

			Par le théorème chinois, on a
			\begin{align}
				K[X] / (X^{2} - X) & \xrightarrow{\sim} K[X] / X \cartprod K[X] / (X -
				1) \\
				P(X) \, mod(X^{2} - X) & \xrightarrow{\sim} (P(X) \, mod(X), P(X), \,
				mod(X - 1))
			\end{align}
			
			De plus, on a vu que $K[X] / X$ est isomorphe à $K$ grace à
			$eval_{0}$ et $K[X] / (X - 1)$ est isomorphe à $K$ grace à
			$eval_{1}$.

			On en déduit que $K[X] / (X^{2} - X)$ est isomorphe à $K \cartprod
			K$.
	\end{itemize}
\end{exemple}

%Le théorème chinois généralisé a une application en théorie des polynômes, dont
%la théorie des anneaux de polynôme sera un peu étudiée au chapitre 2.
%Nous allons maintenant définir l'indicatrice d'Euler, et en donner quelques
%propriétés.
%On s'en servira dans le chapitre 2.

%%	Indicatrice d'Euler
%%		Définition
%%		Propriété
%%		Calcul

\begin{exemple}
	Soient $A = \integer$ et $n \in \integer$.
	Posons $n = p_{1}^{m_{1}} . \cdots . p_{l}^{m_{l}}$ la décomposition de $n$
	en facteurs premiers.

	Posons $I_{k} = p_{k}^{m_{k}} \integer$ pour tout $1 \leq k \leq l$.

	On a, pour tout $1 \leq j, k \leq l$, $j \neq k$
	\begin{equation}
		I_{k} + I_{j} = \integer
	\end{equation}

	De plus,
	\begin{equation}
		ppcm(p_{1}^{m_{1}}, \cdots, p_{l}^{m_{l}}) = n
	\end{equation}
	et
	\begin{equation}
		\bigcap_{1 \leq k \leq l} p_{k}^{m_{k}} \integer = n \integer
	\end{equation}

	On en déduit que

	\begin{equation}
		\integer / n \integer \xrightarrow{\sim} \prod_{1 \leq i \leq l}
		(\integer / p_{i}^{m_{i}} \integer)
	\end{equation}

	et comme corollaire, on obtient

	\begin{equation}
		(\integer / n \integer)^{\cartprod} \xrightarrow{\sim} \prod_{1 \leq i \leq l}
		(\integer / p_{i}^{m_{i}} \integer)^{\cartprod}
	\end{equation}
	vu comme des groupes. On a alors

	\begin{equation}
		\ordergroup{(\integer / n \integer)^{\cartprod}} = \ordergroup{\prod_{1 \leq i \leq l}
		(\integer / p_{i}^{m_{i}} \integer)^{\cartprod}}
	\end{equation}
\end{exemple}

\begin{definition}
	L'indicatrice d'Euler, noté souvent $\phi$ est la fonction

	\begin{equation}
		\GSfunction{\phi}{\naturel}{\naturel} : n \rightarrow
		|(\integer/n\integer)^{\cartprod}|
	\end{equation}
\end{definition}

L'indicatrice d'Euler donne donc, pour chaque $n \in \naturel$, le nombre
d'élément inversible de $\integer/n\integer$.
On obtient directement une première propriété : $\phi(p) = p - 1$ où $p$ est
premier. En effet $\integer/p\integer$ est un corps, donc tous les éléments non
nuls sont inversibles.

Enonçons quelques propriétés de l'indicatrice d'Euler.

\begin{proposition}
	\begin{itemize}
		\item Soit $p$ un nombre premier. Alors $\phi(p) = p - 1$.
		\item Soit $n, m \geq 1$. Alors, si $m$ et $n$ sont premiers entre eux,
			$\phi(n m) = \phi(n) \phi(m)$.
		\item Soit $m \geq 1$ et soit $p$ un nombre premier. Alors $\phi(p^{m})
			= (p - 1) p^{m - 1}$
		\item Soient $n \in \naturel$, et $n = p_{1}^{m_{1}} . \cdots .
			p_{l}^{m_{k}}$ sa décomposition en facteurs premiers. Alors
			$\phi(n) = \displaystyle \prod_{1 \leq i \leq l} (p_{i} - 1)
			p_{i}^{m_{i} - 1}$
		\item Soit $n \geq 1$ un entier. Alors $\phi(n) = \displaystyle \sum_{d | n}
			\phi(d)$.
	\end{itemize}
\end{proposition}

\ifdefined\outputproof
\begin{proof}

\end{proof}
\fi

%\subsection{Anneau euclidien}

%La proposition suivante est très importante pour définir les idéaux de certains
%anneaux commutatifs, appelé anneau euclidien. Malheureusement, tout anneau
%commutatif n'est pas euclidien.

%\begin{definition} [Anneau euclidien]
	%Soit $A$ un anneau. On dit que $A$ est un \textbf{anneau euclidien} si il existe
	%une fonction $\GSfunction{N}{A}{\naturel_{0}}$ tel que : $\forall a, b \in
	%A_{0}$, $\exists q, r \in A$ tel que $a = qb + r$ avec $N(r) < N(b)$ ou r =
	%0.
	%\label{euclidian_ring}
%\end{definition}

%\begin{exemple}
	%\begin{enumerate}
		%\item $\integer$ avec la valeur absolue est un anneau euclidien.
		%\item $\integer[i]$, les entiers de Gauss, muni de la norme complexe
			%usuelle au carré est un anneau euclidien.
	%\end{enumerate}
%\end{exemple}

%\begin{proposition}
	%Tout anneau euclidien est principal, c'est-à-dire que tous ses idéaux sont
	%engendré par un élément.
	%\label{euclidian_implies_principal}
%\end{proposition}

%\ifdefined\outputproof
%\begin{proof}
	%Prenons un anneau euclidien A, ainsi qu'un idéal I de A. Si I = $\GSset{0}$
	%ou $\GSset{0, 1}$, on a fini ($\GSset{0}$ est un idéal engendré par 0, et
	%$\GSset{0, 1} = I$ car $1 \in I$.

	%Sinon, I possède deux éléments x et y non nul. Nous pouvons supposer, sans
	%perte de généralité\footnote{Si y ne vérifie pas la minimalité, on prend z
	%tel que N(z) est minimal, et on remplace dans la preuve x par y et y par
%z.}, que y est tel que N(y) = $\displaystyle min_{a \in I_{0}} \GSset{N(a)}$ (en
%effet, l'ensemble des N(a) est un sous ensemble de N, donc il possède un
%minimum).  On a, comme A est euclidien, qu'il existe q et r appartenant à A tel
%que $x
	%= qy + r$, avec N(r) < N(y) ou r = 0.
	%On a alors $r = x - qy$ avec $x \in I$, $qy \in I$ (comme $q \in A$ et $y
	%\in I$ et I idéal), donc $r \in I$ comme I idéal.

	%Or, si $N(r) < N(y)$ avec r non nul, cela contredirait la minimalité de N(y)
	%car $r \in I_{0}$. Donc r est nul.

	%On a alors que $x = qy$, c'est-à-dire que si on prend à chaque fois deux
	%éléments quelconques de I, l'un est multiple de l'autre, c'est-à-dire que
	%I est principal.
%\end{proof}
%\fi

%\begin{corollary}
	%Les idéaux de $\integer$ (resp $\integer$[i]) sont les n$\integer$ où $n \in
	%\integer$ (resp $(a+bi) \integer$[i] où $a, b \in \integer$).
%\end{corollary}

%On peut également en déduire que si un anneau possède un idéal non principal,
%alors il ne possède pas de division euclidienne, ce qui est le cas pour $Z[X]$.
%On en déduit donc que tout anneau commutatif n'est pas euclidien.

%Par contre, nous avons :
%\begin{proposition}
	%Tout anneau euclidien est commutatif.
%\end{proposition}

%Nous avons la chaine d'inclusion suivante (non-exhaustive) pour les classes
%d'idéaux :
%\begin{theorem}
	%Corps
	%$\displaystyle \subsetneq$ Anneaux euclidiens
	%$\displaystyle \subseteq^{\ref{euclidian_implies_principal}}$ Anneaux principaux
	%$\subseteq$ Anneaux factoriels $\subseteq$ Anneaux intègres $\subseteq$
	%Anneaux commutatifs.
	%\label{ideal_class}
%\end{theorem}

\subsection{Irréductibilité}

\begin{definition} [Élément irréductible]
	Soit $a \in A$. $a$ est dit irréductible si $\forall x, y \in A$, $a = xy
	\Rightarrow x \in A^{*}$ ou $y \in A^{*}$.
	\label{irreductible_element}
\end{definition}

\begin{exemple}
	Soit $n \in \integer$, $n$ est irréductible dans $\integer$ ssi $n = \pm p$
	où p premier.
\end{exemple}

\begin{proposition}
	$a$ est irréductible dans $A$ ssi $\forall u \in A^{*}$, $ua$ est
	irréductible dans $A$.
\end{proposition}

\ifdefined\outputproof
\begin{proof}

\end{proof}
\fi

\begin{proposition}
	Soit $f$ un isomorphisme d'anneau. $a$ est irréductible ssi $f(a)$ est
	irréductible.
\end{proposition}

\ifdefined\outputproof
\begin{proof}

\end{proof}
\fi

\begin{exemple}
	On a vu que la conjugaison est un automorphisme d'anneau, donc si $z$ est
	irréductible, $\conjuguate{z}$ l'est aussi.
\end{exemple}

Une remarque importante est que l'irréductibilité d'un élément dépend de l'anneau
dans lequel nous nous trouvons. On a par exemple $2$ irréductible dans
$\integer$ mais $2$ est réductible dans $\complex$ parce que $2 = (1 + i) (1 -
i)$.

Donnons maintenant quelques équivalences en termes d'idéaux. La définition
\ref{irreductible_element} date du XIXè siècle, la suivante, plus couramment
utilisée actuellement, date du début du XXè siècle.

\begin{proposition}

	$a$ est irréductible

	$\Leftrightarrow$
	$a \notin A^{*}, a \ne 0$ $b | a \Rightarrow b \in A^{*}$ ou $\exists u \in
	A^{*}, b = ua$.

	$\Leftrightarrow$
	$(a) \ne (0)$, $(a) \ne A$, $(a) \subseteq (b) \Rightarrow (b) = A$ ou $(a)
	= (b)$

	$\Leftrightarrow$
	$(a) \ne (0)$, $(a)$ maximal parmi les idéaux principaux différents de $A$.
\end{proposition}

\ifdefined\outputproof
\begin{proof}
	Chaque équivalence est une réécriture.
\end{proof}
\fi



\section{Arithmétique des anneaux}

\begin{definition} [Idéal premier]

\end{definition}

\begin{definition} [Idéal maximal]

\end{definition}

\begin{definition} [Anneau intégre]

\end{definition}

\begin{definition} [Corps]

\end{definition}

\begin{proposition}
	Tout idéal maximal est premier.
\end{proposition}

\ifdefined\outputproof
\begin{proof}

\end{proof}
\fi

\begin{proposition}
	Soit $I$ un idéal d'un anneau commutatif $A$.
	\begin{enumerate}
		\item $I$ est un idéal maximal $\Leftrightarrow A/I$ est un
			corps.
		\item $I$ est un idéal premier $\Leftrightarrow A/I$ est
			intégre.
	\end{enumerate}
\end{proposition}

\ifdefined\outputproof
\begin{proof}

\end{proof}
\fi

\begin{exemple}
	%Z/nZ
\end{exemple}

\begin{theorem}
	Tout idéal est contenu dans un idéal maximal.
\end{theorem}

\ifdefined\outputproof
\begin{proof}

\end{proof}
\fi


\section{Irréductibilité dans $\rational[X]$}

\section{Anneau de polynômes}

\begin{proposition}
	\begin{enumerate}
		\item $A$ intègre $\Leftrightarrow$ $(X)$ idéal premier de $A[X]$
		\item $A$ corps $\Leftrightarrow$ $(X)$ idéal maximal de $A[X]$
	\end{enumerate}
\end{proposition}

\ifdefined\outputproof
\begin{proof}
	Si on prend la fonctions surjective $eval_{0}$, on a un isomorphisme induit
	entre $A$ et $A[X]/(X)$. Comme $(X)$ est un idéal premier, $A[X]/(X)$ est
	intègre. Par l'isomorphisme, $A$ est intègre.
\end{proof}
\fi

\begin{proposition}
	Soit $K$ un corps, alors $K[X]$ possède une division euclidienne. Par
	conséquent, $K[X]$ est principal.
\end{proposition}

\ifdefined\outputproof
\begin{proof}
\end{proof}
\fi

\begin{proposition}
	Soit $K$ un corps,
	Pour tout idéal $I$ non nul, il existe un \textbf{unique} $P \in K[X]$
	\textbf{monique} tel que $I = (P)$.
\end{proposition}

\ifdefined\outputproof
\begin{proof}
	Commençons par l'existence.

	Comme $K$ est un corps, $K[X]$ est principal, et donc chaque idéal est
	engendré par un élément. Notons celui-ci $Q(X) = a_{n}X^{n} + \ldots +
	a_{1}X + a_{0}, a_{n} \ne 0$. On a donc $I = (Q)$. Comme $K$ est un corps, on peut
	définir $a_{n}^{-1}$, et $P(X) = a_{n}^{-1}Q(X) \in I$. Celui-ci est
	monique, et on a de plus que $(P) = (Q) = I$ car $P$ et $Q$ sont copremiers.

	Supposons maintenant qu'il existe un autre polynôme monique $S(X)$
	engendrant $I$. (A finir).
\end{proof}
\fi

\begin{proposition}
	Soit $K$ un corps, $P \in K[X]$.
	\begin{enumerate}
		\item $(P)$ est maximal $\Leftrightarrow$ $P$ est irréductible.
		\item $(P)$ est premier $\Leftrightarrow$ $P = 0$ ou $P$
			irréductible.
	\end{enumerate}
\end{proposition}

\ifdefined\outputproof
\begin{proof}

\end{proof}
\fi

On a alors comme corollaire :

\begin{corollary}
	\begin{enumerate}
		\item $P$ est irréductible $\Leftrightarrow$ $K[X]/(P)$ est un corps.
		\item $P = 0$ ou $P$ est irréductible $\Leftrightarrow$ $K[X]/(P)$ est
			intègre.
	\end{enumerate}
\end{corollary}

\ifdefined\outputproof
\begin{proof}

\end{proof}
\fi

\begin{proposition}
	Soit $K[X]$ où $K$ est un corps. Soit $P \in K[X]$.

	Si $P$ est de degré $1$, alors $P$ est irréductible.

	De plus, si $K$ est algébriquement clos et $P$ irréductible, alors $P$ est
	de degré $1$.
\end{proposition}

\subsection{Polynômes à coefficients dans $\integer[X]$}

Dans cette partie nous allons étudier les propriétés que les polynômes à
coefficients dans $\integer[X]$ possèdent dans $\rational[X]$ et dans
$\integer[X]$.

% Parler des liens avec la théorie des nombres ?
% Donner des applications avec la théorie des nombres ?

Dans la suite, on considère que $P(X) = a_{n}X^{n} + \ldots + a_{1}X + a_{0}$
est un polynôme à coefficients dans $\integer$.

Rappelons d'abord la définition d'irréductibilité dans le cas de $\rational[X]$
et $\integer[X]$.

Prenons $P(X)$ qui n'est pas inversible dans $\rational[X]$ (resp dans
$\integer[X]$).
Ce polynôme $P(X)$ est irréductible dans $\rational[X]$ (resp $\integer[X]$) si
pour toute décomposition de $P(X)$ en deux polynômes $Q(X)$ et $R(X)$ ($P =
QR$), on a $Q(X) \in \rational$ ou $R(X) \in \rational_{0}$ (resp $Q(X) = \pm 1$
ou $R(X) = \pm 1$ car les inversibles de $\integer[X]$ sont $1$ et $-1$).

Prenons $P(X) = 2X-2$. On a $P(X)$ qui est irréductible dans
$\rational[X]$ mais celui-ci est réductible dans $\integer[X]$ car $P(X) =
2(X-1)$. On n'a donc pas ($P(X)$ irréductible dans $\rational[X] \Rightarrow
P(X)$ irréductible dans $\integer[X]$).

Nous avons tout de même, sous certaines hypothèses, que l'implication est vraie.

\begin{proposition}
	Si $pgcd(a_{0}, \ldots, a_{n}) = \pm 1$, alors :

	Si $P(X)$ est irréductible dans $\rational[X]$, alors $P(X)$ est
	irréductible dans $\integer[X]$.
\end{proposition}

\ifdefined\outputproof
\begin{proof}

\end{proof}
\fi

\begin{theorem} [Critère Eisenstein\footnote{Ferdinand \textbf{Gotthold} Max
		\textbf{Eisenstein} : 16 avril 1823 (Berlin) - 11 octobre 1862 (Berlin).
Mathématicien allemand d'origine juive. Mort de tuberculose. Élève de Dirichlet
à l'université de Berlin.}]
	Soit $p$ premier tel que :

	\begin{itemize}
		\item $p$ ne divise pas $a_{n}$
		\item $\forall i \in \left\{ 0, \ldots, n - 1 \right\}$, $p$ divise
			$a_{i}$.
		\item $p^{2}$ ne divise pas $a_{0}$.
	\end{itemize}
	\label{critere_E}
	Alors, $P(X)$ est irréductible dans $\rational[X]$.
\end{theorem}

\ifdefined\outputproof
\begin{proof}

\end{proof}
\fi

Nous avons également un théorème qui permet de déterminer l'irréductibilité d'un
polynôme. Celui-ci se sert du corps $\integer/p\integer$ où p est premier.

\begin{theorem}
	Si il existe un nombre premier $p$ tel que le polynôme $\overline{P}(X) =
	\overline{a_{n}}X^{n} + \ldots + \overline{a_{1}}X + \overline{a_{0}}$ où
	$\overline{a_{i}} = a_{i}$ mod(p) est irréductible dans
	$\integer/p\integer[X]$, alors $P(X)$ est irréductible dans $\rational[X]$.
\end{theorem}

\ifdefined\outputproof
\begin{proof}

\end{proof}
\fi
%exemple

\begin{corollary}
	Grace à ce dernier théorème, on en déduit que $X^{p - 1} + \ldots + X + 1$
	est irréductible dans $Q[X]$ (et dans $\integer[X]$) quand p est premier.
\end{corollary}

\ifdefined\outputproof
\begin{proof}

\end{proof}
\fi

\subsection{Polynômes cyclotomiques}

Nous allons maintenant étudier certains polynômes appelés polynômes
cyclotomiques.

%D'abord, rappelons que le polynôme $X^{n} - 1$ possèdent n racines complexes,
%appelées racines n-ième de l'unité, et qui sont $e^{\frac{2ki\pi}{n}}$, où $k$ va
%de $0$ à $n - 1$. On peut de la même manière dire que $k \in
%\integer/n\integer$.

%De plus, les racines n-ième de l'unité forment un groupe d'ordre n, qui est
%l'unique sous-groupe d'ordre n de $\complex$, qui est cyclique, et isomorphe à
%$\integer/n\integer$.

%On a, si on prend une de ces racines qui est d'ordre $d$, alors il engendre un
%unique sous-groupe d'ordre $d$. Ce $d$ doit diviser $n$ par le théorème de
%Lagrange. Dénotons cet unique sous-groupe d'ordre d par $S_{d}$.


