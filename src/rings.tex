\chapter{Anneaux}

\section{Théorie générale}

\subsection{Morphisme d'anneaux}
\begin{proposition}
	Il existe un unique morphisme d'anneau entre $\rational$ (resp $\integer$)
	et $\complex$.
\end{proposition}

\begin{proof}
	Si on a un morphisme f, on a par récurrence $\forall n \in \integer, f(n) =
	n$. D'où, le seule morphisme est $({Id_{\complex}})_{|\integer}$.

	La démonstration est la même pour $\rational$.
\end{proof}

\begin{proposition}
	Il existe seulement deux morphismes d'anneaux entre $\rational[i]$ (resp
	$\integer[i]$) et $\complex$.
\end{proposition}

\begin{proof}

\end{proof}

\begin{corollary}
	\begin{enumerate}
		\item \GSautomorphismDef{corps}{\rational} =
			\GSautomorphismDef{anneau}{\rational} = \GSset{Id}
		\item \GSautomorphismDef{anneau}{\integer} = \GSset{Id}
	\end{enumerate}
\end{corollary}

\begin{proof}
	En effet, si d'autres automorphismes existeraient, on pourrait étendre
	l'ensemble d'arrivé en $\complex$, et ils resteraient des morphismes.
\end{proof}

\begin{question}
	Combien il y a d'automorphismes de corps sur $\complex$ ? Sur $\real$ ?
	\href{http://www.math.uga.edu/~pete/Kestelman51.pdf}{Ici pour $\complex$}

	Pour $\real$, le seul automorphisme continu est l'identité. En effet, comme
	$\rational$ est dense dans $\real$, on a une suite d'élément de $\rational$
	qui tend vers $x \in \real$. On a alors que $f(x)$ est la limite de la suite
	$f(x_{n})$, où f maintenant est un morphisme de $\rational$ dans $\real$.
	Or, le seul morphisme est l'identité, donc $f(x) = x$ à la limite.

	Pour $\complex$, les seuls automorphismes sont l'identité et la conjugaison
	(même raisonnement que pour $\real$ en décomposant l'image de f en partie
	réelle et partie imaginaire).
\end{question}

\subsection{Idéaux}

\begin{definition} [Idéal]
	
	\label{ideal}
\end{definition}

\begin{definition} [Idéal principal]
	Un idéal $I$ est un \textbf{idéal principal} si il est engendré par un seul
	élément. On note, si $a \in A$ engendre $I$, $I = (a)$.
	\label{principal_ideal}
\end{definition}

Nous pouvons alors créer une classe d'anneaux : les anneaux principaux.

\begin{definition} [Anneau principal]
	Un anneau $A$ est un \textbf{anneau principal} si tout idéal est principal.
	\label{principal_ring}
\end{definition}

\begin{exemple}
	
\end{exemple}

\begin{proposition}
	Soit \GSfunction{f}{A}{B} où A et B sont deux anneaux commutatifs. Alors
	Im(f) est un sous-anneau de B, et ker(f) est un idéal de A.

	De plus on a :
	\begin{enumerate}
		\item Si Im(f) est un idéal, f est surjectif.
		\item Si Ker(f) est un sous-anneau, B = \GSset{0_{B}}.
	\end{enumerate}
\end{proposition}

\begin{proof}
	$1$. On doit montrer que$ Im(f) = B$. Comme $f$ morphisme, $1 \in Im(f)$, on a
	donc que $Im(f) = B$ car $Im(f)$ est un idéal.
	$2$. 
\end{proof}

\begin{proposition}
	Les seuls idéaux d'un corps $\mathbb{K}$ est $(0_{\mathbb{K}})$ où
	$\mathbb{K}$.
\end{proposition}

\begin{proof}
	Soit $I$ un idéal. S'il est nul, on a fini. Sinon on a un élément non
	nul $a \in I$. Comme $K$ est un corps, $a^{-1} \in K$, et donc $a^{-1}a = 1
	\in I$ car $I$ idéal. Donc $I = K$.
\end{proof}

\begin{corollary}
	Soit $\mathbb{K}$ un corps et L un anneau, et
	\GSfunction{f}{$\mathbb{K}$}{L} un morphisme d'anneau. Alors soit L est nul,
	soit f est injectif.
\end{corollary}

\begin{proof}
	
\end{proof}

\begin{proposition}
	Soit I et J deux idéaux d'un anneau commutatifs A tel que $I \subseteq J$.
	Alors J/I est un idéal de A/I.
\end{proposition}

\begin{proof}
	
\end{proof}

\subsection{Anneau euclidien}

La proposition suivante est très importante pour définir les idéaux de certains
anneaux commutatifs, appelé anneau euclidien. Malheureusement, tout anneau
commutatif n'est pas euclidien.

\begin{definition} [Anneau euclidien]
	Soit A un anneau. On dit que A est un \textbf{anneau euclidien} si il existe
	une fonction \GSfunction{N}{A}{$\naturel_{0}$} tel que : $\forall a, b \in
	A_{0}$, $\exists q, r \in A$ tel que $a = qb + r$ avec $N(r) < N(b)$ ou r =
	0.
	\label{euclidian_ring}
\end{definition}

\begin{exemple}
	\begin{enumerate}
		\item $\integer$ avec la valeur absolue est un anneau euclidien.
		\item $\integer$[i], les entiers de Gauss, muni de la norme complexe
			usuelle au carré est un anneau euclidien.
	\end{enumerate}
\end{exemple}

\begin{proposition}
	Tout anneau euclidien est principal, c'est-à-dire que tous ses idéaux sont
	engendré par un élément.
	\label{euclidian_implies_principal}
\end{proposition}

\begin{proof}
	Prenons un anneau euclidien A, ainsi qu'un idéal I de A. Si I = \GSset{0}
	ou \GSset{0, 1}, on a fini (\GSset{0} est un idéal engendré par 0, et
	\GSset{0, 1} = I car $1 \in I$.

	Sinon, I possède deux éléments x et y non nul. Nous pouvons supposer, sans
	perte de généralité\footnote{Si y ne vérifie pas la minimalité, on prend z
	tel que N(z) est minimal, et on remplace dans la preuve x par y et y par
z.}, que y est tel que N(y) = $\displaystyle min_{a \in I_{0}}$\GSset{N(a)} (en
effet, l'ensemble des N(a) est un sous ensemble de N, donc il possède un
minimum).
	On a, comme A est euclidien, qu'il existe q et r appartenant à A tel que $x
	= qy + r$, avec N(r) < N(y) ou r = 0.
	On a alors $r = x - qy$ avec $x \in I$, $qy \in I$ (comme $q \in A$ et $y
	\in I$ et I idéal), donc $r \in I$ comme I idéal.
	
	Or, si $N(r) < N(y)$ avec r non nul, cela contredirait la minimalité de N(y)
	car $r \in I_{0}$. Donc r est nul.

	On a alors que $x = qy$, c'est-à-dire que si on prend à chaque fois deux
	éléments quelconques de I, l'un est multiple de l'autre, c'est-à-dire que
	I est principal.
\end{proof}

\begin{corollary}
	Les idéaux de $\integer$ (resp $\integer$[i]) sont les n$\integer$ où $n \in
	\integer$ (resp $(a+bi) \integer$[i] où $a, b \in \integer$).
\end{corollary}

On peut également en déduire que si un anneau possède un idéal non principal,
alors il ne possède pas de division euclidienne, ce qui est le cas pour $Z[X]$.
On en déduit donc que tout anneau commutatif n'est pas euclidien.

Par contre, nous avons :
\begin{proposition}
	Tout anneau euclidien est commutatif.
\end{proposition}

Nous avons la chaine d'inclusion suivante (non-exhaustive) pour les classes
d'idéaux :
\begin{theorem}
	Corps 
	$\displaystyle \subsetneq$ Anneaux euclidiens
	$\displaystyle \subseteq^{\ref{euclidian_implies_principal}}$ Anneaux principaux
	$\subseteq$ Anneaux factoriels $\subseteq$ Anneaux intègres $\subseteq$
	Anneaux commutatifs.
	\label{ideal_class}
\end{theorem}

\subsection{Irréductibilité}

\begin{definition} [Élément irréductible]
	Soit $a \in A$. $a$ est dit irréductible si $\forall x, y \in A$, $a = xy
	\Rightarrow x \in A^{*}$ ou $y \in A^{*}$.
	\label{irreductible_element}
\end{definition}

\begin{exemple}
	Soit $n \in \integer$, $n$ est irréductible dans $\integer$ ssi $n = \pm p$
	où p premier.
\end{exemple}

\begin{proposition}
	$a$ est irréductible dans $A$ ssi $\forall u \in A^{*}$, $ua$ est
	irréductible dans $A$.
\end{proposition}

\begin{proof}
	
\end{proof}

\begin{proposition}
	Soit $f$ un isomorphisme d'anneau. $a$ est irréductible ssi $f(a)$ est
	irréductible.
\end{proposition}

\begin{proof}
	
\end{proof}

\begin{exemple}
	On a vu que la conjugaison est un automorphisme d'anneau, donc si $z$ est
	irréductible, $\conjugate{z}$ l'est aussi.
\end{exemple}

Une remarque importante est que l'irréductibilité d'un élément dépend de l'anneau
dans lequel nous nous trouvons. On a par exemple $2$ irréductible dans
$\integer$ mais $2$ est réductible dans $\complex$ parce que $2 = (1 + i) (1 -
i)$.

Donnons maintenant quelques équivalences en termes d'idéaux. La définition
\ref{irreductible_element} date du XIXè siècle, la suivante, plus couramment
utilisée actuellement, date du début du XXè siècle.

\begin{proposition}

	$a$ est irréductible

	$\Leftrightarrow$
	$a \notin A^{*}, a \ne 0$ $b | a \Rightarrow b \in A^{*}$ ou $\exists u \in
	A^{*}, b = ua$.

	$\Leftrightarrow$
	$(a) \ne (0)$, $(a) \ne A$, $(a) \subseteq (b) \Rightarrow (b) = A$ ou $(a)
	= (b)$

	$\Leftrightarrow$
	$(a) \ne (0)$, $(a)$ maximal parmi les idéaux principaux différents de $A$.
\end{proposition}

\begin{proof}
	Chaque équivalence est une réécriture.
\end{proof}
\section{Théorème chinois}

\subsection{Aspect historique}
% Aspect historique

\subsection{Énoncé}
% Théorème + preuve
Nous souhaitons, pour deux anneaux donnés, et sous certaines conditions, montrer
que $A/(I \cap J)$ est isomorphe à $A/I \times A/J$.

Soient $I, J$ deux idéaux de $A$. Alors, nous pouvons construire
\GSfunction{$f$}{$A$}{$A/I \times A/J$} : $a \rightarrow (a + I, a + J)$.
Nous avons $ker(f) = I \cap J$.

On a alors un morphisme injectif \GSfunction{$\overline{f}$}{$A/(I \cap
J)$}{$A/I \times A/J$} induit par $f$.

Il nous manque donc une condition, la surjectivité, pour avoir un isomorphisme
entre ces deux anneaux.
On en vient alors au théorème chinois :

\begin{theorem}
	Si $I + J = A$, alors $\overline{f}$ est surjectif, et donc $A/(I \cap
	J)$ est isomorphe à $A/I \times A/J$.
\end{theorem}

\begin{proof}
	
\end{proof}

On peut généraliser ce théorème à un nombre fini d'idéaux $I_{1}, \ldots,
I_{n}$.

On construit comme précédemment $f$, et on déduit un morphisme injectif
$\overline{f}$. Il suffit de trouver une condition pour que $\overline{f}$. Nous
avons alors le résultat suivant, dit théorème chinois généralisé.

\begin{theorem}
	Si $\forall k \in \left\{1, \ldots, n\right\}$, $I_{k} + \displaystyle
	\bigcap_{1 = i \ne k}^{n} I_{i} = A$, alors $\overline{f}$ est surjectif, et
	donc \GSprodSet{i}{1}{n}{A/I} est isomorphe à $\displaystyle A/\bigcap_{i = 1}^{n}
	I_{i}$.
\end{theorem}

\begin{proof}
	
\end{proof}

\subsection{Interprétation et applications}
% Interprétation et applications

Le théorème chinois généralisé a une application en théorie des polynômes, dont
la théorie des anneaux de polynôme sera un peu étudiée au chapitre 2.
Nous allons maintenant définir l'indicatrice d'Euler, et en donner quelques
propriétés.
On s'en servira dans le chapitre 2.

\begin{definition}
	L'indicatrice d'Euler, noté souvent $\phi$ est la fonction
	\GSfunction{$\phi$}{$\naturel$}{$\naturel$} : $n \rightarrow
	|(\integer/n\integer)^{*}|$.
\end{definition}

L'indicatrice d'Euler donne donc, pour chaque $n \in \naturel$, le nombre
d'élément inversible de $\integer/n\integer$.
On obtient directement une première propriété : $\phi(p) = p - 1$ où $p$ est
premier. En effet $\integer/p\integer$ est un corps, donc tous les éléments non
nuls sont inversibles.

%	Indicatrice d'Euler
%		Définition
%		Propriété
%		Calcul


\section{Classification partielle des idéaux}

% Changer le titre ?
% Il en existe d'autre ?
% Peut-on vraiment tous les classer ?

\begin{definition} [Idéal premier]

\end{definition}

\begin{definition} [Idéal maximal]

\end{definition}

\begin{definition} [Anneau intégre]
	
\end{definition}

\begin{definition} [Corps]

\end{definition}

\begin{proposition}
	Tout idéal maximal est premier.
\end{proposition}

\begin{proof}
	
\end{proof}

\begin{proposition}
	Soit $I$ un idéal d'un anneau commutatif $A$.
	\begin{enumerate}
		\item $I$ est un idéal maximal $\Leftrightarrow A/I$ est un
			corps.
		\item $I$ est un idéal premier $\Leftrightarrow A/I$ est
			intégre.
	\end{enumerate}
\end{proposition}

\begin{proof}
	
\end{proof}

\begin{exemple}
	%Z/nZ
\end{exemple}

\begin{theorem}
	Tout idéal est contenu dans un idéal maximal.
\end{theorem}

\begin{proof}
	
\end{proof}
\section{Anneau de polynômes}

\begin{proposition}
	\begin{enumerate}
		\item $A$ intègre $\Leftrightarrow$ $(X)$ idéal premier de $A[X]$
		\item $A$ corps $\Leftrightarrow$ $(X)$ idéal maximal de $A[X]$
	\end{enumerate}
\end{proposition}

\begin{proof}
	Si on prend la fonctions surjective $eval_{0}$, on a un isomorphisme induit
	entre $A$ et $A[X]/(X)$. Comme $(X)$ est un idéal premier, $A[X]/(X)$ est
	intègre. Par l'isomorphisme, $A$ est intègre.
\end{proof}

\begin{proposition}
	Soit $K$ un corps, alors $K[X]$ possède une division euclidienne. Par
	conséquent, $K[X]$ est principal.
\end{proposition}

\begin{proof}
\end{proof}

\begin{proposition}
	Soit $K$ un corps,
	Pour tout idéal $I$ non nul, il existe un \textbf{unique} $P \in K[X]$
	\textbf{monique} tel que $I = (P)$.
\end{proposition}

\begin{proof}
	Commençons par l'existence.

	Comme $K$ est un corps, $K[X]$ est principal, et donc chaque idéal est
	engendré par un élément. Notons celui-ci $Q(X) = a_{n}X^{n} + \ldots +
	a_{1}X + a_{0}, a_{n} \ne 0$. On a donc $I = (Q)$. Comme $K$ est un corps, on peut
	définir $a_{n}^{-1}$, et $P(X) = a_{n}^{-1}Q(X) \in I$. Celui-ci est
	monique, et on a de plus que $(P) = (Q) = I$ car $P$ et $Q$ sont copremiers.
	
	Supposons maintenant qu'il existe un autre polynôme monique $S(X)$
	engendrant $I$. (A finir).
\end{proof}

\begin{proposition}
	Soit $K$ un corps, $P \in K[X]$.
	\begin{enumerate}
		\item $(P)$ est maximal $\Leftrightarrow$ $P$ est irréductible.
		\item $(P)$ est premier $\Leftrightarrow$ $P = 0$ ou $P$
			irréductible.
	\end{enumerate}
\end{proposition}

\begin{proof}
	
\end{proof}

On a alors comme corollaire :

\begin{corollary}
	\begin{enumerate}
		\item $P$ est irréductible $\Leftrightarrow$ $K[X]/(P)$ est un corps.
		\item $P = 0$ ou $P$ est irréductible $\Leftrightarrow$ $K[X]/(P)$ est
			intègre.
	\end{enumerate}
\end{corollary}

\begin{proof}
	
\end{proof}

\begin{proposition}
	Soit $K[X]$ où $K$ est un corps. Soit $P \in K[X]$.
	
	Si $P$ est de degré $1$, alors $P$ est irréductible.
	
	De plus, si $K$ est algébriquement clos et $P$ irréductible, alors $P$ est
	de degré $1$.
\end{proposition}
\subsection{Polynômes à coefficients dans $\integer[X]$}

Dans cette partie nous allons étudier les propriétés que les polynômes à
coefficients dans $\integer[X]$ possèdent dans $\rational[X]$ et dans
$\integer[X]$.

% Parler des liens avec la théorie des nombres ?
% Donner des applications avec la théorie des nombres ?

Dans la suite, on considère que $P(X) = a_{n}X^{n} + \ldots + a_{1}X + a_{0}$
est un polynôme à coefficients dans $\integer$.

Rappelons d'abord la définition d'irréductibilité dans le cas de $\rational[X]$
et $\integer[X]$.

Prenons $P(X)$ qui n'est pas inversible dans $\rational[X]$ (resp dans
$\integer[X]$).
Ce polynôme $P(X)$ est irréductible dans $\rational[X]$ (resp $\integer[X]$) si
pour toute décomposition de $P(X)$ en deux polynômes $Q(X)$ et $R(X)$ ($P =
QR$), on a $Q(X) \in \rational$ ou $R(X) \in \rational_{0}$ (resp $Q(X) = \pm 1$
ou $R(X) = \pm 1$ car les inversibles de $\integer[X]$ sont $1$ et $-1$).

Prenons $P(X) = 2X-2$. On a $P(X)$ qui est irréductible dans
$\rational[X]$ mais celui-ci est réductible dans $\integer[X]$ car $P(X) =
2(X-1)$. On n'a donc pas ($P(X)$ irréductible dans $\rational[X] \Rightarrow
P(X)$ irréductible dans $\integer[X]$).

Nous avons tout de même, sous certaines hypothèses, que l'implication est vraie.

\begin{proposition}
	Si $pgcd(a_{0}, \ldots, a_{n}) = \pm 1$, alors :

	Si $P(X)$ est irréductible dans $\rational[X]$, alors $P(X)$ est
	irréductible dans $\integer[X]$.
\end{proposition}

\begin{proof}
	
\end{proof}

\begin{theorem} [Critère Eisenstein\footnote{Ferdinand \textbf{Gotthold} Max
		\textbf{Eisenstein} : 16 avril 1823 (Berlin) - 11 octobre 1862 (Berlin).
Mathématicien allemand d'origine juive. Mort de tuberculose. Élève de Dirichlet
à l'université de Berlin.}]
	Soit $p$ premier tel que :

	\begin{itemize}
		\item $p$ ne divise pas $a_{n}$
		\item $\forall i \in \left\{ 0, \ldots, n - 1 \right\}$, $p$ divise
			$a_{i}$.
		\item $p^{2}$ ne divise pas $a_{0}$.
	\end{itemize}
	\label{critere_E}
	Alors, $P(X)$ est irréductible dans $\rational[X]$.
\end{theorem}

\begin{proof}
	
\end{proof}

Nous avons également un théorème qui permet de déterminer l'irréductibilité d'un
polynôme. Celui-ci se sert du corps $\integer/p\integer$ où p est premier.

\begin{theorem}
	Si il existe un nombre premier $p$ tel que le polynôme $\overline{P}(X) =
	\overline{a_{n}}X^{n} + \ldots + \overline{a_{1}}X + \overline{a_{0}}$ où
	$\overline{a_{i}} = a_{i}$ mod(p) est irréductible dans
	$\integer/p\integer[X]$, alors $P(X)$ est irréductible dans $\rational[X]$.
\end{theorem}

\begin{proof}
	
\end{proof}
%exemple

\begin{corollary}
	Grace à ce dernier théorème, on en déduit que $X^{p - 1} + \ldots + X + 1$
	est irréductible dans $Q[X]$ (et dans $\integer[X]$) quand p est premier.
\end{corollary}

\begin{proof}
	
\end{proof}
\subsection{Polynômes cyclotomiques}

Nous allons maintenant étudier certains polynômes appelés polynômes
cyclotomiques.

%D'abord, rappelons que le polynôme $X^{n} - 1$ possèdent n racines complexes,
%appelées racines n-ième de l'unité, et qui sont $e^{\frac{2ki\pi}{n}}$, où $k$ va
%de $0$ à $n - 1$. On peut de la même manière dire que $k \in
%\integer/n\integer$.

%De plus, les racines n-ième de l'unité forment un groupe d'ordre n, qui est
%l'unique sous-groupe d'ordre n de $\complex$, qui est cyclique, et isomorphe à
%$\integer/n\integer$.

%On a, si on prend une de ces racines qui est d'ordre $d$, alors il engendre un
%unique sous-groupe d'ordre $d$. Ce $d$ doit diviser $n$ par le théorème de
%Lagrange. Dénotons cet unique sous-groupe d'ordre d par $S_{d}$.


