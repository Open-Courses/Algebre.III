\chapter{Anneaux}

\section{Théorie générale}

\subsection{Morphisme d'anneaux}
\begin{proposition}
	Il existe un unique morphisme d'anneau entre $\rational$ (resp $\integer$)
	et $\complex$.
\end{proposition}

\begin{proof}
	Si on a un morphisme f, on a par récurrence $\forall n \in \integer, f(n) =
	n$. D'où, le seule morphisme est $({Id_{\complex}})_{|\integer}$.

	La démonstration est la même pour $\rational$.
\end{proof}

\begin{proposition}
	Il existe seulement deux morphismes d'anneaux entre $\rational[i]$ (resp
	$\integer[i]$) et $\complex$.
\end{proposition}

\begin{proof}

\end{proof}

\begin{corollary}
	\begin{enumerate}
		\item \GSautomorphismDef{corps}{\rational} =
			\GSautomorphismDef{anneau}{\rational} = \GSset{Id}
		\item \GSautomorphismDef{anneau}{\integer} = \GSset{Id}
	\end{enumerate}
\end{corollary}

\begin{proof}
	En effet, si d'autres automorphismes existeraient, on pourrait étendre
	l'ensemble d'arrivé en $\complex$, et ils resteraient des morphismes.
\end{proof}

\begin{question}
	Combien il y a d'automorphismes de corps sur $\complex$ ? Sur $\real$ ?
	\href{http://www.math.uga.edu/~pete/Kestelman51.pdf}{Ici pour $\complex$}

	Pour $\real$, le seul automorphisme continu est l'identité. En effet, comme
	$\rational$ est dense dans $\real$, on a une suite d'élément de $\rational$
	qui tend vers $x \in \real$. On a alors que $f(x)$ est la limite de la suite
	$f(x_{n})$, où f maintenant est un morphisme de $\rational$ dans $\real$.
	Or, le seul morphisme est l'identité, donc $f(x) = x$ à la limite.

	Pour $\complex$, les seuls automorphismes sont l'identité et la conjugaison
	(même raisonnement que pour $\real$ en décomposant l'image de f en partie
	réelle et partie imaginaire).
\end{question}

\subsection{Idéaux}

\begin{definition} [Idéal]
	
\end{definition}

\begin{exemple}
	
\end{exemple}

\begin{proposition}
	Soit \GSfunction{f}{A}{B} où A et B sont deux anneaux commutatifs. Alors
	Im(f) est un sous-anneau de B, et ker(f) est un idéal de A.

	De plus on a :
	\begin{enumerate}
		\item Si Im(f) est un idéal, f est surjectif.
		\item Si Ker(f) est un sous-anneau, B = \GSset{0_{B}}.
	\end{enumerate}
\end{proposition}

\begin{proof}
	
\end{proof}

\begin{proposition}
	Les seuls idéaux d'un corps $\mathbb{K}$ est $(0_{\mathbb{K}})$ où
	$\mathbb{K}$.
\end{proposition}

\begin{proof}
	
\end{proof}

\begin{corollary}
	Soit $\mathbb{K}$ un corps et L un anneau, et
	\GSfunction{f}{$\mathbb{K}$}{L} un morphisme d'anneau. Alors soit L est nul,
	soit f est injectif.
\end{corollary}

\begin{proof}
	
\end{proof}

\begin{proposition}
	Soit I et J deux idéaux d'un anneau commutatifs A tel que $I \subseteq J$.
	Alors J/I est un idéal de A/I.
\end{proposition}

\begin{proof}
	
\end{proof}

\begin{definition}
	Soit A un anneau. On dit que A est euclidien si il existe une fonction
	\GSfunction{N}{A}{$\naturel_{0}$} tel que : $\forall a, b \in A_{0}$, $\exists q,
	r \in A$ tel que $a = qb + r$ avec $N(r) < N(b)$ ou r = 0.
\end{definition}

\begin{exemple}
	\begin{enumerate}
		\item $\integer$ avec la valeur absolue est un anneau euclidien.
		\item $\integer$[i], les entiers de Gauss, muni de la norme complexe
			usuelle au carré est un anneau euclidien.
	\end{enumerate}
\end{exemple}

\subsection{Anneau euclidien}

La proposition suivante est très importante pour définir les idéaux d'un anneau
commutatif.

\begin{definition} [Anneau euclidien]
	
\end{definition}

\begin{proposition}
	Tout anneau euclidien est principal, c'est-à-dire que tous ses idéaux sont
	engendré par un élément.
\end{proposition}

\begin{proof}
	Prenons un anneau euclidien A, ainsi qu'un idéal I de A. Si I = \GSset{0}
	ou \GSset{0, 1}, on a fini (\GSset{0} est un idéal engendré par 0, et
	\GSset{0, 1} = I car $1 \in I$.

	Sinon, I possède deux éléments x et y non nul. Nous pouvons supposer, sans
	perte de généralité\footnote{Si y ne vérifie pas la minimalité, on prend z
	tel que N(z) est minimal, et on remplace dans la preuve x par y et y par
z.}, que y est tel que N(y) = $\displaystyle min_{a \in I_{0}}$\GSset{N(a)} (en
effet, l'ensemble des N(a) est un sous ensemble de N, donc il possède un
minimum).
	On a, comme A est euclidien, qu'il existe q et r appartenant à A tel que $x
	= qy + r$, avec N(r) < N(y) ou r = 0.
	On a alors $r = x - qy$ avec $x \in I$, $qy \in I$ (comme $q \in A$ et $y
	\in I$ et I idéal), donc $r \in I$ comme I idéal.
	
	Or, si $N(r) < N(y)$ avec r non nul, cela contredirait la minimalité de N(y)
	car $r \in I_{0}$. Donc r est nul.

	On a alors que $x = qy$, c'est-à-dire que si on prend à chaque fois deux
	éléments quelconques de I, l'un est multiple de l'autre, c'est-à-dire que
	I est principal.
\end{proof}

\begin{corollary}
	Les idéaux de $\integer$ (resp $\integer$[i]) sont les n$\integer$ où $n \in
	\integer$ (resp $(a+bi) \integer$[i] où $a, b \in \integer$).
\end{corollary}

On peut également en déduire que si un anneau possède un idéal non principal,
alors il ne possède pas de division euclidienne, ce qui est le cas pour $Z[X]$.


\subsection{Irréductibilité}

\begin{definition} [Élément irréductible]
	
\end{definition}

\begin{exemple}
	Soit $n \in \integer$, $n$ est irréductible dans $\integer$ ssi n est
	premier.
\end{exemple}

\begin{proposition}
	$a$ est irréductible dans $A$ ssi $\forall u \in A^{*}$, $ua$ est
	irréductible dans $A$.
\end{proposition}

\begin{proof}
	
\end{proof}

\begin{proposition}
	Soit $f$ un isomorphisme d'anneau. $a$ est irréductible ssi $f(a)$ est
	irréductible.
\end{proposition}

\begin{proof}
	
\end{proof}

\begin{exemple}
	On a vu que la conjugaison est un automorphisme d'anneau, donc si $z$ est
	irréductible, $\conjugate{z}$ l'est aussi.
\end{exemple}

Une remarque importante est que l'irréductibilité d'un élément dépend de l'anneau
dans lequel nous nous trouvons. On a par exemple $2$ irréductible dans
$\integer$ mais $2$ est réductible dans $\complex$ parce que $2 = (1 + i) (1 -
i)$.



\section{Théorème chinois}

\subsection{Aspect historique}
% Aspect historique

\subsection{Énoncé}
% Théorème + preuve

\subsection{Interprétation et applications}
% Interprétation et applications
%	Indicatrice d'Euler
%		Définition
%		Propriété
%		Calcul


\section{Classification partielle des idéaux}

% Changer le titre ?
% Il en existe d'autre ?
% Peut-on vraiment tous les classer ?

\begin{definition} [Idéal premier]

\end{definition}

\begin{definition} [Idéal maximal]

\end{definition}

\begin{definition} [Anneau intégre]
	
\end{definition}

\begin{definition} [Corps]

\end{definition}

\begin{proposition}
	Tout idéal maximal est premier.
\end{proposition}

\begin{proof}
	
\end{proof}

\begin{proposition}
	Soit $I$ un idéal d'un anneau commutatif $A$.
	\begin{enumerate}
		\item $I$ est un idéal maximal $\Leftrightarrow A/I$ est un
			corps.
		\item $I$ est un idéal premier $\Leftrightarrow A/I$ est
			intégre.
	\end{enumerate}
\end{proposition}

\begin{proof}
	
\end{proof}

\begin{exemple}
	%Z/nZ
\end{exemple}

\begin{theorem}
	Tout idéal est contenu dans un idéal maximal.
\end{theorem}

\begin{proof}
	
\end{proof}
\section{Anneau de polynomes}

\begin{proposition}
	Soit $K$ un corps, alors $K[X]$ possède une division euclidienne. Par
	conséquent, $K[X]$ est principal.
\end{proposition}

\begin{proof}
	
\end{proof}

\begin{proposition}
	Soit $K$ un corps,
	$\forall I$ idéal non nul, $\exists ! P \in K[X]$ monique tel que $I = (P)$.
\end{proposition}

\begin{proof}
	
\end{proof}

\begin{proposition}
	Soit $K$ un corps, $P \in K[X]$.
	\begin{enumerate}
		\item $(P)$ est maximal $\Leftrightarrow$ $P$ est irréductible.
		\item $(P)$ est premier $\Leftrightarrow$ $P = 0$ ou $P$
			irréductible.
	\end{enumerate}
\end{proposition}

\begin{proof}
	
\end{proof}

On a alors comme corollaire :

\begin{corollary}
	\begin{enumerate}
		\item $P$ est irréductible $\Leftrightarrow$ $K[X]/(P)$ est un corps.
		\item $P = 0$ ou $P$ est irréductible $\Leftrightarrow$ $K[X]/(P)$ est
			intègre.
	\end{enumerate}
\end{corollary}

\begin{proof}
	
\end{proof}
