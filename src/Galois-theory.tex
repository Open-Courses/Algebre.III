\chapter{Théorie de Galois}

\section{Extension de corps}

Rappelons qu'un \textbf{corps} est un anneau commutatif $A$ tel que le seul
élément non inversible est $0_{A}$.

\begin{definition} [Morphisme de corps]
	\label{def:morphisme_corps}
	Soit $K$ et $F$ deux corps. \textbf{Un morphisme de corps entre $K$ et $F$}
	est un morphisme d'annau.
\end{definition}

\begin{proposition}
	Tout morphisme de corps est injectif
\end{proposition}

\begin{definition} [Extension de corps et sous corps]
	\label{def:extension_sous_corps}
	Soit $K$ et $L$ deux corps. On dit que $L$ est \textbf{une extension de
	corps}, et $K$ \textbf{un sous corps de $L$} si $K \subseteq L$ et on note
	$L/K$.
\end{definition}

\begin{proposition}
	Soit une extension de corps $L/K$. Alors $L$ est un $K$-espace vectoriel.
\end{proposition}

\begin{definition} [Degré d'une extension]
	\label{def:degre_extension}
	Soit $L/K$ une extension de corps. On définit \textbf{le degré de
	l'extension $L/K$} par la dimension de $L$ en tant que $K$-espace
	vectoriel, et on note $\extensionDegree{L}{K}$.
\end{definition}

\begin{definition} [Extension finie]
	Soit $L/K$ une extension de corps. On dit que $L/K$ est \textbf{une
	extension finie} si le degré de $L/K$ ($\extensionDegree{L}{K}$) est fini.
\end{definition}

\begin{remarque}
	Soient $K \subseteq L \subseteq M$ trois corps.
	Alors:
	\begin{enumerate}
		\item $\extensionDegree{L}{K} \leq \extensionDegree{M}{K}$.
		\item $\extensionDegree{M}{K}$ fini $\implies$ $\extensionDegree{L}{K}$
			fini.
		\item $\extensionDegree{L}{K} = 1$ $\equiv$ $L = K$.
	\end{enumerate}
\end{remarque}

\begin{proposition} [Multiplicativité des degrés]
	Soient $K \subseteq L \subseteq M$ trois corps. Alors
	$\extensionDegree{M}{L} \extensionDegree{L}{K} = \extensionDegree{M}{K}$.
\end{proposition}

\begin{proposition} [Multiplicativité des degrés généralisée]
	Soient $L_{1} \subseteq L_{2} \subseteq \cdots \subseteq L_{n}$.
	Alors $\extensionDegree{L_{n}}{L_{1}} =
	\displaystyle \prod_{i = 1}^{n - 1} \extensionDegree{L_{i + 1}}{L_{i}}$.
\end{proposition}

\begin{remarque}
	Soit $K \subseteq L \subseteq M$. Alors $\extensionDegree{L}{K} /
	\extensionDegree{M}{K}$ et $\extensionDegree{M}{L} /
	\extensionDegree{M}{K}$. En particulier, si $\extensionDegree{M}{K}$ est un
	nombre premier, alors il n'existe pas de corps strictement entre $K$ et $M$.
\end{remarque}

\begin{exercice}
	Il n'y a pas de corps strictement compris entre $\complex$ et $\real$.
\end{exercice}

\begin{definition}
	Soient $L/K$ et $M/K$ deux extensions de corps. Alors on définit:
	% Definition de L \inter M
	% Definition de LM
\end{definition}

\begin{definition}
	Soit $L/K$ une extension de corps. Soit $S \subseteq L$ (sans nécessairement
	de structures).

	On définit $K(S)$ par \ldots.
\end{definition}

\begin{proposition}
	Soit $L/K$ une extension de corps. Soit $\alpha$, $\beta \in L$.
	Alors $K(\alpha, \beta) = K(\alpha)K(\beta)$.
\end{proposition}

Prenons $\alpha \in L$ qui est algébrique sur $K$, et construisons le morphisme
d'évaluation $eval_{\alpha, K} : K[X] \rightarrow L : P(X) \rightarrow P(\alpha)$
non nul. On pose $K[\alpha] = Im(eval_{\alpha, K})$.

Comme $K$ est un corps, on a que $K[X]$ est euclidien, et donc que ses idéaux
sont engendrés par un élément.

Comme $eval_{\alpha, K}$ morphisme, on a $\ker{(ev_{\alpha, K})} = (P)$.
% l'évaluation est non nul => ker = (P).

\begin{definition} [Algébrique / transcendant]
	Soit $L/K$ et $\alpha \in L$. on dit que \textbf{$\alpha$ est algébrique sur
	$K$} si $\ker{(eval_{\alpha, K})} \neq \GSset{0}$.
	Sinon, $\alpha$ est dit \textbf{transcendant}.
\end{definition}

\begin{proposition}
	Soit $\alpha \in L$ algèbrique sur $K$. Alors $K[\alpha]$ est un corps.
\end{proposition}

\begin{definition} [Polynome minimal]
	Soit $\alpha \in L$ algébrique sur $K$. Le polynome minimal de $\alpha$ sur
	$K$ est l'unique $P_{\alpha, K} \in K[X]$ monique tel que $\ker{(eval_{\alpha,
	K})} = (P_{\alpha, K})$. En particulier, $P_{\alpha, K}$ est irréductible
	sur $K$.
\end{definition}

\begin{proposition}
	\label{prop:extension_finie_alpha_equiv_alpha_algebrique}
	Soient $L/K$ une extension de corps, et $\alpha \in L$ algébrique sur $K$.
	Soit $n = deg(P_{\alpha, K})$.
	Alors $(1, \alpha, \alpha^{2}, \cdots, \alpha^{n - 1})$ est une base de $K(\alpha)$ en tant que $K$
	espace vectoriel. Et donc, en particulier, $\extensionDegree{K(\alpha)}{K} =
	deg(P_{\alpha, K})$.
\end{proposition}

\begin{corollary}
	Soit $L/K$ une extension de corps, et soit $\alpha \in L$. Alors
	$K(\alpha)/K$ est finie ssi $\alpha$ est algébrique sur $K$.
\end{corollary}

\section{Extension algébrique}

\begin{definition} [Extension algébrique]
	Soit $L/K$ est une extension de corps. On dit que l'extension $L/K$ est
	\textbf{algébrique} si tout élément de $L$ est algébrique sur $K$. De
	manière équivalente, grace à
	\ref{prop:extension_finie_alpha_equiv_alpha_algebrique}, que $K(\alpha)/K$
	est une extension finie pour tout $\alpha$ dans $L$.
\end{definition}

\begin{exemple}
	$\complex/\real$ est une extension algébrique.

	$\rational(i)/\rational$ est une extension algébrique.

	$\real/\rational$ n'est pas une extension algébrique.
\end{exemple}

\begin{proposition}
	Soit $K \subseteq L \subseteq M$. On a que $M/K$ est une extension
	algébrique ssi $M/L$ et $L/K$ sont des extensions algébriques.
\end{proposition}

\begin{proposition}
	Soit $L/K$ une extension finie. Alors $L/K$ est une extension algébrique.
\end{proposition}

\begin{remarque}
	La réciproque est fausse.
\end{remarque}

\begin{proposition}
	Soit $L/K$ une extension finie. Alors il existe $n \geq 1$, et $\alpha_{1}$,
	\ldots, $\alpha_{n}$ algébrique sur $K$ tel que $L = K(\alpha_{1}, \cdots,
	\alpha_{n})$.
\end{proposition}

\begin{proposition}
	Soit $L/K$ une extension de corps.
	Soit $F$ l'ensemble des éléments de $L$ algébrique sur $K$.
	Alors $F$ est un sous corps de $L$ contenant $K$.
\end{proposition}

\begin{definition}
	On appelle $F$, défini précédemment, \textbf{la cloture algébrique de $K$
	sur $L$}.
	On note habituellement $F$ par $\overline{K}$.
\end{definition}

Remarquons qu'a priori la cloture algébrique est dépendante d'une extension de corps. On
montrera par après que 
\begin{theorem}
	Soit $K$ un corps. Alors il existe un corps algébriquement clos $\Omega$
	contenant $K$.
\end{theorem}

\section{$K$-plongement}

\section{Groupe de Galois en caractéristique nulle}
