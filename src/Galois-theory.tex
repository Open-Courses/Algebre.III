\chapter{Théorie de Galois}

\section{Extension de corps}

Rappelons qu'un \textbf{corps} est un anneau commutatif $A$ tel que le seul
élément non inversible est $0_{A}$.

\begin{definition} [Morphisme de corps]
	\label{def:morphisme_corps}
	Soient $K$ et $F$ deux corps. \textbf{Un morphisme de corps entre $K$ et $F$}
	est un morphisme d'anneau.
\end{definition}

\begin{proposition}
	Tout morphisme de corps est injectif
\end{proposition}

\begin{definition} [Extension de corps et sous corps]
	\label{def:extension_sous_corps}
	Soient $K$ et $L$ deux corps. On dit que $L$ est \textbf{une extension de
	corps}, et $K$ \textbf{un sous corps de $L$} si $K \subseteq L$ et on note
	$L/K$.
\end{definition}

\begin{proposition}
	Soit une extension de corps $L/K$. Alors $L$ est un $K$-espace vectoriel.
\end{proposition}

\begin{definition} [Degré d'une extension]
	\label{def:degre_extension}
	Soit $L/K$ une extension de corps. On définit \textbf{le degré de
	l'extension $L/K$} par la dimension de $L$ en tant que $K$-espace
	vectoriel, et on note $\extensionDegree{L}{K}$.
\end{definition}

\begin{definition} [Extension finie]
	Soit $L/K$ une extension de corps. On dit que $L/K$ est \textbf{une
	extension finie} si le degré de $L/K$ ($\extensionDegree{L}{K}$) est fini.
\end{definition}

\begin{remarque}
	Soient $K \subseteq L \subseteq M$ trois corps.
	Alors:
	\begin{enumerate}
		\item $\extensionDegree{L}{K} \leq \extensionDegree{M}{K}$.
		\item $\extensionDegree{M}{K}$ fini $\implies$ $\extensionDegree{L}{K}$
			fini.
		\item $\extensionDegree{L}{K} = 1$ $\equiv$ $L = K$.
	\end{enumerate}
\end{remarque}

\begin{proposition} [Multiplicativité des degrés]
	Soient $K \subseteq L \subseteq M$ trois corps. Alors
	$\extensionDegree{M}{L} \extensionDegree{L}{K} = \extensionDegree{M}{K}$.
\end{proposition}

\begin{proposition} [Multiplicativité des degrés généralisée]
	Soient $L_{1} \subseteq L_{2} \subseteq \cdots \subseteq L_{n}$.
	Alors $\extensionDegree{L_{n}}{L_{1}} =
	\displaystyle \prod_{i = 1}^{n - 1} \extensionDegree{L_{i + 1}}{L_{i}}$.
\end{proposition}

\begin{remarque}
	Soit $K \subseteq L \subseteq M$. Alors $\extensionDegree{L}{K}$ divise
	$\extensionDegree{M}{K}$ et $\extensionDegree{M}{L}$ divise
	$\extensionDegree{M}{K}$. En particulier, si $\extensionDegree{M}{K}$ est un
	nombre premier, alors il n'existe pas de corps strictement compris entre $K$
	et $M$.
\end{remarque}

\begin{exercice}
	Il n'y a pas de corps strictement compris entre $\complex$ et $\real$.
\end{exercice}

\begin{definition}
	Soient $L/K$ et $M/K$ deux extensions de corps tel que $L \subseteq E$ et $M
	\subseteq E$ où $E$ est un corps. Alors on définit:

	\begin{itemize}
		\item $LM = \displaystyle \bigcap_{\substack{F \subseteq E
				\text{ corps} \\ M \subseteq F \\ L \subseteq F}}
	F$. C'est la plus grande extension de $K$ contenant $L$ et $M$.
		\item $L \inter M$ est la plus grande extension de $K$ contenue dans $L$
			et $M$. De manière générale, on peut étudier une intersection
			quelconque d'extension.
	\end{itemize}
\end{definition}

\begin{exercice}
	Si $pgcd(\extensionDegree{L}{K}, \extensionDegree{M}{K}) = 1$. Alors $L
	\inter M = K$.
\end{exercice}

\begin{definition}
	Soit $L/K$ une extension de corps. Soit $S$ un sous-ensemble de $L$ (il n'y
	a pas nécessairement de structures sur $S$).

	On définit $K(S)$ par:

	\begin{equation*}
		K(S) = \displaystyle \bigcap_{\substack{K \subseteq F \subseteq L
			\text{ corps} \\ S \subseteq F}} F
	\end{equation*}

	En particulier, quand $S = \GSset{\alpha_{1}, \cdots, \alpha_{n}}$, on note
	$K(S)$ par $K(\alpha_{1}, \cdots, \alpha_{n})$.

	C'est le plus petit corps contenant le corps $K$ et le sous-ensemble $S$.
\end{definition}

\begin{proposition}
	Soit $L/K$ une extension de corps. Soient $\alpha$, $\beta \in L$.
	Alors $K(\alpha, \beta) = K(\alpha)K(\beta)$.
\end{proposition}

\ifdefined\outputproof
\begin{proof}

\end{proof}
\fi

Soit $L/K$ une extension de corps. Prenons $\alpha \in L$, et construisons le
morphisme d'évaluation $eval_{\alpha, K} : K[X] \rightarrow L : P(X) \rightarrow
P(\alpha)$ non nul. On pose $K[\alpha] = Im(eval_{\alpha, K})$.

Comme $K$ est un corps, on a que $K[X]$ est euclidien, donc ses idéaux
sont engendrés par un élément.

Comme $eval_{\alpha, K}$ morphisme, on a $\ker{(ev_{\alpha, K})} = (P)$ car
le noyau de tout morphisme d'anneau est un idéal.

\begin{definition} [Algébrique / transcendant]
	Soit $L/K$ et $\alpha \in L$. on dit que \textbf{$\alpha$ est algébrique sur
	$K$} si $\ker{(eval_{\alpha, K})} \neq \GSset{0}$.
	Sinon, $\alpha$ est dit \textbf{transcendant}.
\end{definition}

\begin{proposition}
	Soit $\alpha \in L$ algèbrique sur $K$. Alors $K[\alpha]$ est un corps.
	En particulier, $K(\alpha) = K[\alpha]$.
\end{proposition}

\ifdefined\outputproof
\begin{proof}

\end{proof}
\fi

\begin{definition} [Polynome minimal]
	Soit $\alpha \in L$ algébrique sur $K$. \textbf{Le polynome minimal de $\alpha$ sur
	$K$} est l'unique $P_{\alpha, K} \in K[X]$ monique tel que $\ker{(eval_{\alpha,
	K})} = (P_{\alpha, K})$. En particulier, $P_{\alpha, K}$ est irréductible
	sur $K$.
\end{definition}

\begin{proposition}
	\label{prop:extension_finie_alpha_equiv_alpha_algebrique}
	Soient $L/K$ une extension de corps, et $\alpha \in L$ algébrique sur $K$.
	Soit $n = deg(P_{\alpha, K})$.
	Alors $(1, \alpha, \alpha^{2}, \cdots, \alpha^{n - 1})$ est une base de $K(\alpha)$ en tant que $K$
	espace vectoriel. Et donc, en particulier, $\extensionDegree{K(\alpha)}{K} =
	deg(P_{\alpha, K})$.
\end{proposition}

\ifdefined\outputproof
\begin{proof}

\end{proof}
\fi

\begin{corollary}
	Soit $L/K$ une extension de corps, et soit $\alpha \in L$. Alors
	$K(\alpha)/K$ est finie ssi $\alpha$ est algébrique sur $K$.
\end{corollary}

\ifdefined\outputproof
\begin{proof}

\end{proof}
\fi

\section{Extension algébrique}

\begin{definition} [Extension algébrique]
	Soit $L/K$ est une extension de corps. On dit que l'extension $L/K$ est
	\textbf{algébrique} si tout élément de $L$ est algébrique sur $K$. De
	manière équivalente, grace à
	\ref{prop:extension_finie_alpha_equiv_alpha_algebrique}, que $K(\alpha)/K$
	est une extension finie pour tout $\alpha$ dans $L$.
\end{definition}

\begin{exemple}
	$\complex/\real$ est une extension algébrique.

	$\rational(i)/\rational$ est une extension algébrique.

	$\real/\rational$ n'est pas une extension algébrique.
\end{exemple}

\begin{proposition}
	Soit $K \subseteq L \subseteq M$. On a que $M/K$ est une extension
	algébrique ssi $M/L$ et $L/K$ sont des extensions algébriques.
\end{proposition}

\ifdefined\outputproof
\begin{proof}

\end{proof}
\fi

\begin{proposition}
	Soit $L/K$ une extension finie. Alors $L/K$ est une extension algébrique.
\end{proposition}

\ifdefined\outputproof
\begin{proof}

\end{proof}
\fi

\begin{remarque}
	La réciproque est fausse.
\end{remarque}

\begin{proposition}
	Soit $L/K$ une extension finie. Alors il existe $n \geq 1$, et $\alpha_{1}$,
	\ldots, $\alpha_{n}$ algébriques sur $K$ tel que $L = K(\alpha_{1}, \cdots,
	\alpha_{n})$.
\end{proposition}

\ifdefined\outputproof
\begin{proof}

\end{proof}
\fi

\begin{proposition}
	Soit $L/K$ une extension de corps.
	Soit $F$ l'ensemble des éléments de $L$ algébriques sur $K$.
	Alors $F$ est un sous corps de $L$ contenant $K$.
\end{proposition}

\ifdefined\outputproof
\begin{proof}

\end{proof}
\fi

\begin{definition}
	On appelle $F$, défini précédemment, \textbf{la cloture algébrique de $K$
	sur $L$}.
	On note habituellement $F$ par $\overline{K}$.
\end{definition}

Remarquons qu'a priori la cloture algébrique est dépendante d'une extension de
corps. On montrera par après qu'en réalité, si on prend deux clotures
algébriques, les théories sur celles-ci sont les mêmes. On pourra donc choisir
notre cloture algébrique 'préférée'.

\begin{theorem}
	Soit $K$ un corps. Alors il existe un corps algébriquement clos $\Omega$
	contenant $K$.
\end{theorem}

\ifdefined\outputproof
\begin{proof}

\end{proof}
\fi

\begin{lemma}
	Soit $L/K$ une extension de corps. Soient $\Omega$ algébriquement clos
	contenant $K$, et
	$\GSfunction{\sigma}{K}{\Omega}$ le morphisme d'inclusion.

	Soit $\alpha \in L$ algébrique sur $K$.

	Alors il existe un plongement $\GSfunction{\tau}{K(\alpha)}{\Omega}$ tel que
	$\tau_{|K} = \sigma$.
\end{lemma}

\ifdefined\outputproof
\begin{proof}

\end{proof}
\fi

\begin{theorem} [Extension des plongements]
	Soit $L/K$ algébrique.
	Soient $\Omega$ algébriquement clos contenant $K$ et $\GSfunction{\sigma}{K}{\Omega}$ le
	morphisme d'inclusion.
	Alors il existe $\GSfunction{\tau}{L}{\Omega}$ plongement tel que $\tau_{|K}
	= \sigma$.
\end{theorem}

\ifdefined\outputproof
\begin{proof}

\end{proof}
\fi

\begin{corollary}
	Soient $K$ corps, $\Omega_{1}$ et $\Omega_{2}$ algébriquement clos contenant
	$K$.

	Soit $F_{1}$ (resp. $F_{2}$) la cloture algébrique de $K$ dans $\Omega_{1}$
	(resp. dans $\Omega_{2}$).

	Alors il existe un isomorphisme $K$-linéaire entre $F_{1}$ et
	$F_{2}$. En d'autres termes, $F_{1}$ et $F_{2}$ sont isomorphes.
\end{corollary}

\ifdefined\outputproof
\begin{proof}

\end{proof}
\fi

On en conclut que si on veut étudier les extensions algébriques de $K$, il
suffit de choisir un corps algébriquement clos $\Omega$ contenant $K$, et
d'étudier $\overline{K}$, la cloture algébrique de $K$ dans $\Omega$. Par la
suite, nous dirons que nous prenons une cloture algébrique de $K$.

\section{$K$-plongement}

Soit $K$ un corps de caractéristique nulle (voir
\ref{subsection:ring_caracteristic}).

Fixons une cloture algébrique $\overline{K}$ de $K$.

\begin{definition}
	Soit $L/K$ algébrique.
	\textbf{Un $K$-plongement} est un morphisme de corps
	$\GSfunction{\sigma}{L}{\overline{K}}$ $K$-linéaire.
\end{definition}

Remarquons que nous avons $L \subseteq \overline{K}$ car $L$ algébrique, et
$\overline{K}$ contient tous les éléments algébriques sur $K$.

Un $K$-plongement de corps $\GSfunction{\sigma}{L}{\overline{K}}$ est un
$K$-plongement ssi $\sigma$ fixe tous les éléments $a$ de $K$, ie $\sigma{a} =
a$.
On a donc $\sigma_{|K} : K \rightarrow \overline{K}$ qui est le morphisme
d'inclusion de $K$ dans $\overline{K}$.

Rappelons que si $\alpha$ est algébrique, alors $K(\alpha)$ est un corps
contenant $K$.

\begin{proposition}
	Soient $\alpha \in \overline{K}$, et
	$\GSfunction{\sigma}{K(\alpha)}{\overline{K}}$.

	Alors $\sigma(P(\alpha)) = P(\sigma(\alpha))$
\end{proposition}

\ifdefined\outputproof
\begin{proof}

\end{proof}
\fi

En conclusion, un $K$-plongement de $K(\alpha)$ dans $\overline{K}$ est
uniquement déterminé par l'image de $\alpha$.

Par un même raisonnement, si on prend $\alpha_{1}, \cdots, \alpha_{n}$ et un
$K$-plongement de $K(\alpha_{1}, \cdots, \alpha_{n})$ dans $\overline{K}$, alors
il suffit de connaitre les $\sigma(\alpha_{i})$ pour $1 \leq i \leq n$.

Prenons maintenant le cas du polynome minimal $P_{\alpha, K}$.
On a, par définition, $P_{\alpha, K}(\alpha) = 0$.

On obtient alors la proposition suivante.

\begin{proposition}
	Soit $P_{\alpha, K}$ le polynome minimal de $\alpha$ sur $K$.

	Soit $\GSfunction{\sigma}{K(\alpha)}{\overline{K}}$ un $K$-plongement. Alors
	$\sigma(\alpha)$ est racine de $P_{\alpha, K}$.

	En particulier, si on pose $N$ le nombre de racines de $P_{\alpha, K}$,
	alors il y a au plus $K$-plongement de $K(\alpha)$ dans $\overline{K}$.
\end{proposition}

\ifdefined\outputproof
\begin{proof}

\end{proof}
\fi

\begin{definition}
	Soit $L/K$ algébrique.

	On définit l'ensemble $\plongement{K}{L}{\overline{K}} =
	\GSset{\GSfunction{\sigma}{L}{\overline{K}}, \text{$K$-plongement}}$
\end{definition}

\begin{exemple}
	$\cardinal{\plongement{\rational}{\rational(\sqrt[3]{2})}{\rational}} \leq 3$
\end{exemple}

% Page 26 et 27 pas faites.

\begin{proposition}
	Soient $F/K$ une extension algébrique, et $\alpha \in \overline{K}$.
	Soit $\GSfunction{\sigma}{F}{\overline{K}}$ un $K$-plongement ($\sigma \in
	\plongement{K}{F}{\overline{K}}$).

	Alors l'application:

	\begin{align*}
		& \GSsetDef{\tau \in \plongement{K}{F(\alpha)}{\overline{K}}}{\tau_{|F} =
		\sigma} \rightarrow \GSset{\text{racines dans $\overline{K}$ de
		$\sigma(P_{\alpha, F})$}} \\
		& \tau &\rightarrow& \tau(\alpha)
	\end{align*}

	est bijective. Nous venons donc de faire le lien entre les plongements et
	les racines du polynome minimal.
\end{proposition}

\ifdefined\outputproof
\begin{proof}

\end{proof}
\fi

\begin{exemple}
	On sait que
	$\plongement{\rational}{\rational(\sqrt[4]{2})}{\overline{\rational}}$
	comporte au plus $4$ éléments distincts. Notons les $\sigma_{0}$, $\sigma_{1}$,
	$\sigma_{2}$ et $\sigma_{3}$,
	Nous avons $P_{\sqrt[4]{2}, \rational}(X) = X^{4} - 1$.

	$P_{\sqrt[4]{2}, \rational}(X)$ possèdant $4$ racines distinctes, la
	proposition nous dit alors que nous avons $4$
	$K$-plongements de $\rational{\sqrt[4]{2}}$ dans $\overline{\rational}$, et
	ces plongements sont donnés par
	$\GSfunction{\sigma_{k}}{\rational(\sqrt[4]{2})}{\overline{\rational}}$ :
	$\sqrt[4]{2} \rightarrow \zeta^{k}_{4} \sqrt[4]{2}$ pour $0 \leq k \leq 3$.
\end{exemple}

\begin{proposition}
	Rappelons que nous supposons que $K$ est de caractéristique nulle.

	Soient $F/K$ une extension algébrique et $P(X)$ irréductible dans $F[X]$.
	Alors toutes les racines de $P(X)$ dans $\overline{K}$ sont simples.
\end{proposition}

\ifdefined\outputproof
\begin{proof}

\end{proof}
\fi

\begin{corollary}
	Soient $F/K$ algébrique et $\alpha \in \overline{K}$.
	Soit $\sigma \in \plongement{K}{F}{\overline{K}}$.

	Alors $\cardinal{\GSsetDef{\tau \in \plongement{K}{F(\alpha)}{\overline{K}}}{\tau_{|K} =
\sigma}} = \extensionDegree{F(\alpha)}{F}$.
\end{corollary}

\ifdefined\outputproof
\begin{proof}

\end{proof}
\fi

\begin{proposition}
	Soit $L/K$ une extension finie.

	Alors $\cardinal{\plongement{K}{L}{\overline{K}}} = \extensionDegree{L}{K}$.
\end{proposition}

\ifdefined\outputproof
\begin{proof}

\end{proof}
\fi

% exemple page 31 et 32


\begin{theorem} [de l'élément primitif]
	\label{theorem:primitif_element}
	Soit $K$ un corps de caractéristique nulle.
	Soit $L/K$ une extension finie. Alors il existe $\alpha \in L$ tel que $L =
	K(\alpha)$.
\end{theorem}

\ifdefined\outputproof
\begin{proof}

\end{proof}
\fi

\begin{exercice}
	Montrer que $\rational(\sqrt{2}, \sqrt{3}) = \rational(\sqrt{2} + \sqrt{3})$
	et que $\rational(\zeta_{3}, \zeta_{2}) = \rational(\zeta_{3} \zeta_{2})$.
\end{exercice}

\begin{remarque}
	Le théorème de l'élément primitif \ref{theorem:primitif_element} est aussi
	valable pour les corps finis.
\end{remarque}

\section{Groupe de Galois en caractéristique nulle}

\begin{definition} [Groupe de Galois]
	Soit $L/K$ une extension finie.

	On définit \textbf{le groupe de Galois de l'extension $L/K$}:
	\begin{align*}
		\galoisGroup{L}{K} & := \GSautomorphismDef{K}{L} \\
		& = \GSsetDef{\GSfunction{\tau}{L}{L}}{\text{$\tau$ isomorphisme
		$K$-linéaire de corps}}
	\end{align*}

	$(\galoisGroup{L}{K}, \circ)$ est un groupe.
\end{definition}

Soit $\GSfunction{i}{L}{\overline{K}}$ le morphisme d'injection de $L$ dans
$\overline{K}$.

Alors l'application

\begin{center}
$
\begin{aligned}
	\galoisGroup{L}{K} &\rightarrow& \plongement{K}{L}{\overline{K}} \\
	\sigma &\rightarrow& i \circ \sigma
\end{aligned}
$
\end{center}

est injective.

Soit $\GSfunction{\tau}{L}{\overline{K}}$ $K$-plongmenent tel que $\tau(L)
\subseteq L$. Comme $L/K$ est de dimension finie, on a par le théorème du rang
que $\tau(L) = L$
D'où $\tau \in \galoisGroup{L}{K}$.

On identifie donc, grace à cette injection,

\begin{equation*}
	\galoisGroup{L}{K} = \GSsetDef{\sigma \in
	\plongement{K}{L}{\overline{K}}}{\sigma(L) \subseteq L}
\end{equation*}

\begin{exemple}
	Soit $L = K(\alpha)$, $\alpha \in \overline{K}$.

	Alors
	\begin{align*}
		\galoisGroup{K(\alpha)}{K}
		& = \GSsetDef{\sigma \in
			\plongement{K}{K(\alpha)}{\overline{K}}}{\sigma(K(\alpha)) \subseteq
			K(\alpha)} \\
		& = \GSsetDef{\sigma \in
			\plongement{K}{K(\alpha)}{\overline{K}}}{\sigma(\alpha) \in
			K(\alpha)} \\
		& \isomorph Rac(P_{\alpha, K}) \inter K(\alpha)
	\end{align*}

	En particulier, $\ordergroup{\galoisGroup{L}{K}} =
	\cardinal{\plongement{K}{L}{\overline{K}}} = \extensionDegree{L}{K}$.
\end{exemple}

\begin{exemple} [Exercice]
	\begin{itemize}
		\item $\galoisGroup{K}{K} = \GSset{Id_{K}}$.
		\item $\galoisGroup{\rational(\sqrt{2})}{\rational} =
			\generatedGroup{\sigma} \isomorph
			\integer/2\integer$ où $\sigma(\sqrt{2}) = - \sqrt{2}$.
		\item $\galoisGroup{\rational(\sqrt[3]{2})}{\rational} =
			\GSset{Id_{\rational}}$
		\item $\galoisGroup{\rational(\sqrt[4]{2})}{\rational} \isomorph
			\integer/2\integer$ avec $\sigma(\sqrt[4]{2}) = - \sqrt[4]{2}$
		\item $\galoisGroup{\rational(\zeta_{3}, \sqrt[3]{2})}{\rational}
			\isomorph S_{3}$
		\item $\galoisGroup{\rational(\zeta_{4}, \sqrt[4]{2})}{\rational}
			\isomorph D_{4}$
	\end{itemize}
\end{exemple}

Nous en venons à la définition d'extension galoisienne.

\begin{definition}
	Soit $L/K$ une extension finie.
	$L/K$ est \textbf{(une extension finie) galoisienne} si $\galoisGroup{L}{K} =
	\plongement{K}{L}{\overline{K}}$
\end{definition}

C'est-à-dire que:

\begin{align*}
	L/K \text{ galoisienne}
	& \equiv \forall \sigma \in \plongement{K}{L}{\overline{K}}, \sigma(L) \subseteq L \\
	& \equiv \ordergroup{\galoisGroup{L}{K}} = \extensionDegree{L}{K}
\end{align*}

\begin{proposition}
	Prenons maintenant $L = K(\alpha)$ avec $\alpha \in \overline{K}$.

	Alors $K(\alpha)$ galoisienne $\equiv Rac(P_{\alpha, K}) \subseteq K(\alpha)$.
\end{proposition}

\ifdefined\outputproof
\begin{proof}

\end{proof}
\fi

\begin{exemple} [Exercice]
	\begin{itemize}
		\item $K/K$ est galoisienne.
		\item $\rational(\sqrt{2})/\rational$ est galoisienne.
		\item $\rational(\zeta_{3}, \sqrt[3]{2})/\rational$ galoisienne.
		\item $\rational(\zeta_{4}, \sqrt[4]{2})/\rational$ galoisienne.
		\item $\rational(\sqrt[3]{2})/\rational$ n'est pas galoisienne.
		\item $\rational(\sqrt[4]{2})/\rational$ n'est pas galoisienne.
	\end{itemize}
\end{exemple}

Remarquons que nous avons $\rational \subseteq \rational(\sqrt{2}) \subseteq
\rational(\sqrt[4]{2})$ avec $\rational(\sqrt{2})/\rational$ et
$\rational(\sqrt[4]{2})/\rational(\sqrt{2})$ galoisiennes.
\textbf{Or, $\rational(\sqrt[4]{2})/\rational$ n'est pas galoisienne}. La
propriété d'être galoisienne n'est pas transitive !

\begin{definition}
	Soit $P(X) \in K[X]$ tel que $n := deg(P) \geq 1$. Soit $Rac(P(X)) :=
	\GSset{\alpha_{1}, \cdots, \alpha_{n}}$ l'ensemble
	des racines de $P(X)$ dans $\overline{K}$.

	On appelle $K(\alpha_{1}, \cdots, \alpha_{n}) = K(Rac(P(X)))$ \textbf{le
	corps de décomposition de $P(X)$}.
\end{definition}

\begin{remarque}
	Soit $P(X) = \displaystyle \prod_{i = 1}{d} (X - \alpha_{i})^{m_{i}}$ et
	$P_{0}(X) = \displaystyle \prod_{i = 1}{d} (X - \alpha_{i})$. Alors $P(X)$
	et $P_{0}(X)$ ont le même corps de décomposition.
\end{remarque}

\begin{exemple}
	\begin{itemize}
		\item $\rational(\sqrt{2})$ est le corps de décomposition de $X^{2} -
			2$.
		\item $\rational(\sqrt[3]{2}, \zeta_{3})$ est le corps de décomposition
			de $X^{3} - 2$.
		\item $\rational(\sqrt[4]{2}, \zeta_{4})$ est le corps de décomposition
			de $X^{4} - 2$.
		\item $\rational(\zeta_{n})$ est le corps de décomposition de $X^{n} -
			1$.
	\end{itemize}
\end{exemple}

\begin{proposition}
	Soit $L/K$ finie. Alors $L/K$ est galoisienne ssi $L$ est le corps de
	décomposition d'un polynome de $K[X]$.
\end{proposition}

\ifdefined\outputproof
\begin{proof}

\end{proof}
\fi

Maintenant, nous allons étudier les sous-groupes de $\galoisGroup{L}{K}$.
Commençons d'abord par définir des objets grace aux sous-ensemble du groupe de
Galois de $L/K$.

\begin{definition}
	Soit $L/K$ une extension de corps.
	Soit $S \subseteq \galoisGroup{L}{K}$ un sous-ensemble fini.

	On pose $L^{S} := \GSsetDef{x \in L}{\forall \sigma \in S, \sigma(x) = x}$.
	$L^{S}$ comprend tous les éléments fixes par les éléments de $S$.
\end{definition}

Montrons maintenant quelques propriétés.

\begin{proposition}
	\begin{itemize}
		\item $L^{S}$ est un sous-corps de $L$ contenant $K$.
		\item $S \subseteq T \implies L^{T} \subseteq L^{S}$ (décroissance)
		\item $L^{S} = L^{\generatedGroup{S}}$ où $\generatedGroup{S}$ est le
			sous-groupe engendré par $S$.
	\end{itemize}
\end{proposition}

\ifdefined\outputproof
\begin{proof}

\end{proof}
\fi

La dernière proposition nous montre qu'il nous suffit d'étudier les sous-groupes
de $\galoisGroup{L}{K}$ pour déterminer tous les $L^{S}$ où $S$ est un
sous-ensemble de $\galoisGroup{L}{K}$.

Donnons alors des propriétés quand $S$ est un sous-groupe.

\begin{proposition}
	Soit $H$ et $H'$ deux sous-groupes de $\galoisGroup{L}{K}$. Alors:
	\begin{itemize}
		\item $L^{HH'} = L^{H} \inter L^{H'}$
		\item $L^{H \inter H'} = L^{H} L^{H'}$
	\end{itemize}
\end{proposition}

\ifdefined\outputproof
\begin{proof}

\end{proof}
\fi

\section{La correspondance de Galois}

Soit $L/K$ une extension finie galoisienne. Soit $\alpha \in L$.
Soit $\sigma \in \galoisGroup{L}{K}$.
Soit $\beta \in Rac(P_{\alpha, K})$.
Alors on a:

$P_{\alpha, K}(\beta) = 0 \implies P_{\alpha, K}(\sigma(\beta)) = 0$

Donc, le groupe $\galoisGroup{L}{K}$ agit sur tous les éléments de
$Rac(P_{\alpha, K})$.

C'est-à-dire que l'application
$\GSfunction{\gamma}{\galoisGroup{L}{K}}{Rac(P_{\alpha, K})}$ est bien définie
et est une action de groupe.

\begin{proposition}
	L'action $\gamma$ est transitive, c'est-à-dire que $Rac(P_{\alpha, K}) =
	\GSsetDef{\sigma(\alpha)}{\sigma \in \galoisGroup{L}{K}}$.
\end{proposition}

\ifdefined\outputproof
\begin{proof}

\end{proof}
\fi

On obtient alors $P_{\alpha, K}(X) = \displaystyle \prod_{\sigma \in
	\galoisGroup{L}{K}} (X - \sigma(\alpha)) = \displaystyle \prod_{\sigma \in
		\plongement{K}{L}{\overline{K}}} (X - \sigma(\alpha))$

\begin{theorem}
	Soit $L/K$ une extension finie galoisienne.

	Alors $L^{\galoisGroup{L}{K}} = K$
\end{theorem}

\ifdefined\outputproof
\begin{proof}

\end{proof}
\fi

\begin{proposition}
	Soit $L/K$ une extension finie.
	Soit $F$ corps tel que $K \subseteq F \subseteq L$.

	Alors on a $\galoisGroup{L}{F} < \galoisGroup{L}{K}$ (décroissance entre
	sous-corps de $L$ contenant $K$ et sous-groupe de $\galoisGroup{L}{K}$).
\end{proposition}

\ifdefined\outputproof
\begin{proof}

\end{proof}
\fi

\begin{proposition}
	Soit $L/K$ une extension galoisienne, et soit $F$ corps tel que $K \subseteq
	F \subseteq L$.

	Alors $L/F$ est une extension galoisienne.
\end{proposition}

\ifdefined\outputproof
\begin{proof}

\end{proof}
\fi

Remarquons que $F/K$ n'est pas nécessairement galoisienne si $L/K$ est
galoisienne. Un contre-exemple est donné par $\rational(\zeta_{4},
\sqrt[4]{2})$, $\rational(\sqrt[4]{2})$ et $\rational$.

Nous avons alors montré que nous pouvons construire une fonction $\phi$ qui à
chaque sous-corps de $L$ contenant $K$ associe un sous-groupe de
$\galoisGroup{L}{K}$ et inversément, on peut construire une fonction $\psi$ qui
à chaque sous-groupe de $\galoisGroup{L}{K}$, on peut associer un sous-corps de
$L$ contenant $K$.

Formellement, on a:

\begin{center}
	$
	\begin{aligned}
		\phi :
			\GSsetDef{F}{F \text{ corps et } K \subseteq F \subseteq L}
			&\rightarrow \GSsetDef{H}{H < \galoisGroup{L}{K}}
			\\
			F &\rightarrow \galoisGroup{L}{F}
	\end{aligned}
	$

	$
	\begin{aligned}
		\psi :
			\GSsetDef{H}{H < \galoisGroup{L}{K}}
			&\rightarrow
			\GSsetDef{F}{F \text{ corps et } K \subseteq F \subseteq L}
			\\
			H &\rightarrow L^{H}
	\end{aligned}
	$
\end{center}

Quels sont les liens entre $\phi$ et $\psi$ ?
Cette étude va nous mener à la \textit{correspondance de Galois}.

\begin{theorem}
	\label{theorem:galois_correspondance}
	L'application décroissante $\phi$ est bijective et $\psi = \phi^{-1}$.
	C'est-à-dire $\phi \circ \psi = Id$ et $\psi \circ \phi = Id$.
\end{theorem}

\ifdefined\outputproof
\begin{proof}

\end{proof}
\fi

\textbf{Nous venons donc de faire un lien entre les corps intermédiaires entre $L$ et
$K$ tel que $L/K$ est galoisienne, et les sous-groupes de groupe de Galois
$\galoisGroup{L}{K}$.}

Nous avons vu que si nous avons une extension galoisienne $L/K$ et $F$ corps tel
que $K \subseteq F \subseteq L$, alors $L/F$ est galoisienne, mais pas
nécessairement $F/K$.
Nous sommes près à donner une condition nécessaire et suffisante pour que $F/K$
soit galoisienne.

\begin{proposition}
	Soient $L/K$ une extension finie galoisienne. Soit $F$ corps tel que $K
	\subseteq F \subseteq L$.

	Alors $\galoisGroup{L}{F} \lhd \galoisGroup{L}{K}$ ssi $F/K$ est
	galoisienne.

	Dans ce cas, le morphisme de restriction:
	\begin{align*}
		\galoisGroup{L}{K} &\rightarrow \galoisGroup{F}{K} \\
		\sigma &\rightarrow \sigma_{|F}
	\end{align*}

	induit un isomorphisme entre $\galoisGroup{L}{K}/\galoisGroup{L}{F}$ et
	$\galoisGroup{F}{K}$.
\end{proposition}

\ifdefined\outputproof
\begin{proof}

\end{proof}
\fi
