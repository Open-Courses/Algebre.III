\usepackage{amsmath}

%-------------------------------------------------------------------------------
\newcommand{\naturel}{\mathbb{N}}
\newcommand{\integer}{\mathbb{Z}}
\newcommand{\rational}{\mathbb{Q}}
\renewcommand{\real}{\mathbb{R}} %Already defined in physics package
\newcommand{\complex}{\mathbb{C}}
%-------------------------------------------------------------------------------

%Logical
\def\implies{\Rightarrow}
\def\equiv{\Leftrightarrow}
%-------------------------------------------------------------------------------

%Set theory
\def\union{\cup} %Union
\def\inter{\cap} %Intersection
\newcommand{\comp}[1]{#1^{c}} %Complementary
\def\cartprod{\cross}
\newcommand{\cardinal}[1]{\#(#1)}
%-------------------------------------------------------------------------------

%Topology
\def\interior{\mathring}
\def\adh{\overline}
%-------------------------------------------------------------------------------

%Algebra

% Group theory

\newcommand{\ordergroup}[1]{|#1|}
\newcommand{\GSautomorphismDef}[2]{Aut_{\text{#1}}(#2)}

% Field theory

%extensionDegree
\newcommand{\extensionDegree}[2]{[#1, #2]}

%plongement
\newcommand{\plongement}[3]{Hom_{#1}(#2, #3)}

% Spectrum Theory
\newcommand{\spectrum}[1]{\sigma(#1)}
\newcommand{\resolvant}[1]{\rho(#1)}

\def\Ldeux{\mathcal{L}^{2}}
\def\Ldeuxstar{(\mathcal{L}^{2})^{*}}

%GSsequence :
%		#1 : represention of elements of the sequences
%		#2 : indices
%		#3 : set definition
\newcommand{\GSsequence}[3]{(#1_{#2})_{#2 \in #3}}

%GSsetDef :
%		#1 : set elements
\newcommand{\GSset}[1]{\left\{ #1 \right\}}

%GSsetDef :
%		#1 : global set
%		#2 : condition
\newcommand{\GSsetDef}[2]{\left\{#1 \, | \, #2 \right\}}

%GSprodSet :
%		#1 : indice
%		#2 : begin indice
%		#3 : end indice
%		#4 : set
\newcommand{\GSprodSet}[4]{\displaystyle \prod_{#1 = #2}^{#3} #4_{#1}}

%GSsum :
%		#1 : indice
%		#2 : begin indice
%		#3 : end indice
%		#4 : element
\newcommand{\GSsum}[4]{\displaystyle \sum_{#1 = #2}^{#3} #4}

\newcommand{\GSintervalCC}[2]{\left[#1, #2\right]}
%-------------------------------------------------------------------------------

%Analysis :

% conjuguate
\def\conjuguate{\overline}
%GSApplication :
%       #1 : name funtion
%       #2 : begin set
%       #3 : end set
\newcommand{\GSfunction}[3]{#1 : #2 \rightarrow #3}

%GSnorme : Deprecated --> \norm
%		#1 : elements which norme is applied on
\newcommand{\GSnorme}[1]{\norm{#1}}

%GSnormeDef :
%		#1 : elements which norme is applied on
%		#2 : norme indice
\newcommand{\GSnormeDef}[2]{\norm{#1}_{#2}}

%GSnormedSpace :
%		#1 : vectorial space
%		#2 : \GSnorme[Def] with dot as element.
\newcommand{\GSnormedSpace}[2]{(#1, #2)}

%GSdual
%		#1 : vectorial space
\newcommand{\GSdual}[1]{#1^{*}}

%GSbidual
%		#1 : vectorial space
\newcommand{\GSbidual}[1]{#1^{**}}

\newcommand{\GSunitBoule}[1]{\mathcal{B}_{#1}}
\newcommand{\GSclosedUnitBoule}[1]{\adh{\GSunitBoule{#1}}}

\newcommand{\GSweakTopo}[1]{\sigma(#1, #1^{*})}
\newcommand{\GSpreweakTopo}[1]{\sigma(#1^{*}, #1)}

%GSendomorphism
\newcommand{\GSendomorphism}[1]{End(#1)}

%GShomomorphisme
\newcommand{\GShomomorphisme}[2]{Hom(#1, #2)}

%GScontinueEndo
\newcommand{\GScontinueEndo}[1]{\mathcal{L}(#1)}

%\GScontinueHomo
\newcommand{\GScontinueHomo}[2]{\mathcal{L}(#1, #2)}

%\GScompactEndo
\newcommand{\GScompactEndo}[1]{\mathcal{K}(#1)}

%\GScompactHomo
\newcommand{\GScompactHomo}[2]{\mathcal{K}(#1, #2)}

\newcommand{\GSfiniteRankHomo}[2]{\mathcal{R}_{f}(#1, #2)}
\newcommand{\GSfiniteRankEndo}[1]{\mathcal{R}_{f}(#1)}

\newcommand{\GSisomorphism}[1]{Isom(#1)}
\newcommand{\GSisomorphismHomo}[2]{Isom(#1, #2)}
\newcommand{\GSisometryEndo}[1]{Isom(#1)}
%-------------------------------------------------------------------------------

%Model theory

\def\La{\mathcal{L}}
\def\Th{\mathcal{T}}
\def\SA{\mathcal{A}}
\def\SB{\mathcal{B}}

%Ultraproduct
%	1 : indice elements
%	2 : set which contains indices
%	3 : ultrafilter
%	4 : models represention
\newcommand{\GSultraproduct}[4]{\displaystyle {\prod_{#1 \in #2}}^{#3}#4_{#1}}

%Ultrapower
%	1 : indice elements
%	2 : set which contains indices
%	3 : ultrafilter
%	4 : model
\newcommand{\GSultrapower}[4]{\displaystyle {\prod_{#1 \in #2}}^{#3} #4}

%Substructures
\newcommand{\GSsubstructure}[2]{#1 \subseteq #2}

%Elementary Substructures.
\newcommand{\GSelemSubstructure}[2]{#1 \preceq #2}

%Elementary equivalent structures
\newcommand{\GSelemEquivStructure}[3]{#2 \equiv_{#1} #3}
%-------------------------------------------------------------------------------

%Hilbert space
\def\Hilbert{\mathcal{H}}
\newcommand{\GSortho}[1]{#1^{\perp}}
\def\GSid{\cong}
\newcommand{\dotprod}[2]{\bra{#1}\ket{#2}}
\newcommand{\adjointe}[1]{#1^{*}}
%-------------------------------------------------------------------------------

%Group representions
\newcommand{\GSrepr}[2]{Repr(#1, #2)}
\newcommand{\GSreprf}[2]{Repr_{f}(#1, #2)}
\newcommand{\GSrepri}[2]{Repr_{i}(#1, #2)}
%-------------------------------------------------------------------------------
