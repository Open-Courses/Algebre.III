%        File: main.tex
%     Created: Son Dez 21 04:00  2014 C
% Last Change: Son Dez 21 04:00  2014 C
%
\documentclass[a4paper, 12pt]{article}

\usepackage[french]{babel}
\usepackage[T1]{fontenc}
\usepackage[utf8]{inputenc}

\usepackage{hyperref}
\usepackage{amsmath}
\usepackage{amsthm}
\usepackage{amssymb}

\usepackage{amsmath}

\newcommand{\naturel}{\mathbb{N}}
\newcommand{\integer}{\mathbb{Z}}
\newcommand{\rational}{\mathbb{Q}}
\newcommand{\real}{\mathbb{R}}
\newcommand{\complex}{\mathbb{C}}

\let\conjugate\overline

%GSsequence :
%		#1 : represention of elements of the sequences
%		#2 : indices
%		#3 : set definition
\newcommand{\GSsequence}[3]{$(#1_{#2})_{#2 \in #3}$}

%GSset :
%		#1 : global set
\newcommand{\GSset}[1]{$\left\{#1\right\}$}

%GSsetDef :
%		#1 : global set
%		#2 : condition
\newcommand{\GSsetDef}[2]{$\left\{#1 \, | \, #2 \right\}$}

%GSprodSet :
%		#1 : indice
%		#2 : begin indice
%		#3 : end indice
%		#4 : set
\newcommand{\GSprodSet}[4]{$\displaystyle \prod_{#1 = #2}^{#3} #4_{#1}$}

%GSsum :
%		#1 : indice
%		#2 : begin indice
%		#3 : end indice
%		#4 : element
\newcommand{\GSsum}[4]{$\displaystyle \sum_{#1 = #2}^{#3}$ #4}

\newcommand{\GSintervalCC}[2]{$\left[#1, #2\right]$}

%Analysis :

%GSApplication :
%       #1 : name funtion
%       #2 : begin set
%       #3 : end set
\newcommand{\GSfunction}[3]{#1 : #2 $\rightarrow$ #3}

%GSnorme :
%		#1 : elements which norme is applied on
\newcommand{\GSnorme}[1]{$||#1||$}

%GSnormeDef :
%		#1 : elements which norme is applied on

%GSnormedSpace :
%		#1 : vectorial space
%		#2 : \GSnorme[Def] with dot as element.
\newcommand{\GSnormedSpace}[2]{$($#1, #2$)$}

%GSdual
%		#1 : vectorial space
\newcommand{\GSdual}[1]{#1^{*}}

%GSbidual
%		#1 : vectorial space
\newcommand{\GSbidual}[1]{#1^{**}}

%GSendomorphism
\newcommand{\GSendomorphism}[1]{End(#1)}

%GShomomorphisme
\newcommand{\GShomomorphisme}[2]{Hom(#1, #2)}

%GScontinueEndo
\newcommand{\GScontinueEndo}[1]{$\mathcal{L}(#1)$}

%GScontinueHomo
\newcommand{\GScontinueHomo}[2]{$\mathcal{L}(#1, #2)$}

%GSautomorphism
\newcommand{\GSautomorphism}[1]{Aut(#1)}

%GSautomorphismDef
\newcommand{\GSautomorphismDef}[2]{$Aut_{#1}({#2})$}

\usepackage{amsfonts}
\usepackage{amssymb}
\usepackage{amsmath}
\usepackage{amsthm}
\usepackage{mathrsfs}

\newtheorem{definition}{Définition}[section]

\newtheorem{proposition}[definition]{Proposition}
\newtheorem{lemma}[definition]{Lemme}
\newtheorem{corollary}[definition]{Corollaire}
\newtheorem{theorem}[definition]{Théorème}

\newtheorem{exemple}{Exemple}[section]
\newtheorem*{question}{Questions}
\newtheorem*{remarque}{Remarque}

\newtheorem{exercice}{Exercice}[section]


\title{Algèbre III : Anneaux, polynôme et théorie de Galois}

\begin{document}

\maketitle

\begin{abstract}
	Résumé du cours d'algèbre III. 2014 - 2015. \textbf{En cours de rédaction}.
\end{abstract}

\section{Anneaux}

\begin{proposition}
	Il existe un unique morphisme d'anneau entre $\rational$ (resp $\integer$)
	et $\complex$.
\end{proposition}

\begin{proof}
	Si on a un morphisme f, on a par récurrence $\forall n \in \integer, f(n) =
	n$. D'où, le seule morphisme est $({Id_{\complex}})_{|\integer}$.

	La démonstration est la même pour $\rational$.
\end{proof}

\begin{proposition}
	Il existe seulement deux morphismes d'anneaux entre $\rational[i]$ (resp
	$\integer[i]$) et $\complex$.
\end{proposition}

\begin{proof}
\end{proof}

\begin{corollary}
	\begin{enumerate}
		\item \GSautomorphismDef{corps}{\rational} =
			\GSautomorphismDef{anneau}{\rational} = \GSset{Id}
		\item \GSautomorphismDef{anneau}{\integer} = \GSset{Id}
	\end{enumerate}
	\begin{proof}
	En effet, si d'autres automorphismes existeraient, on pourrait étendre
	l'ensemble d'arrivé en $\complex$, et ils resteraient des morphismes.
	\end{proof}
\end{corollary}

\begin{question}
	Combien il y a d'automorphismes de corps sur $\complex$ ? Sur $\real$ ?
	\href{http://www.math.uga.edu/~pete/Kestelman51.pdf}{Ici pour $\complex$}

	Pour $\real$, le seul automorphisme continu est l'identité. En effet, comme
	$\rational$ est dense dans $\real$, on a une suite d'élément de $\rational$
	qui tend vers $x \in \real$. On a alors que $f(x)$ est la limite de la suite
	$f(x_{n})$, où f maintenant est un morphisme de $\rational$ dans $\real$.
	Or, le seul morphisme est l'identité, donc $f(x) = x$ à la limite.

	Pour $\complex$, les seuls automorphismes sont l'identité et la conjugaison
	(même raisonnement que pour $\real$ en décomposant l'image de f en partie
	réelle et partie imaginaire).
\end{question}

\begin{proposition}
	Soit \GSfunction{f}{A}{B} où A et B sont deux anneaux commutatifs. Alors
	Im(f) est un sous-anneau de B, et ker(f) est un idéal de A.

	De plus on a :
	\begin{enumerate}
		\item Si Im(f) est un idéal, f est surjectif.
		\item Si Ker(f) est un sous-anneau, B = \GSset{0_{B}}.
	\end{enumerate}
\end{proposition}

\begin{proposition}
	Les seuls idéaux d'un corps $\mathbb{K}$ est $(0_{\mathbb{K}})$ où
	$\mathbb{K}$.
\end{proposition}

\begin{corollary}
	Soit $\mathbb{K}$ un corps et L un anneau, et
	\GSfunction{f}{$\mathbb{K}$}{L} un morphisme d'anneau. Alors soit L est nul,
	soit f est injectif.
\end{corollary}

\begin{proposition}
	Soit I et J deux idéaux d'un anneau commutatifs A tel que $I \subseteq J$.
	Alors J/I est un idéal de A/I.
\end{proposition}

\begin{definition}
	Soit A un anneau. On dit que A est euclidien si il existe une fonction
	\GSfunction{N}{A}{$\naturel_{0}$} tel que : $\forall a, b \in A_{0}$, $\exists q,
	r \in A$ tel que $a = qb + r$ avec $N(r) < N(b)$ ou r = 0.
\end{definition}

\begin{exemple}
	\begin{enumerate}
		\item $\integer$ avec la valeur absolue est un anneau euclidien.
		\item $\integer$[i], les entiers de Gauss, muni de la norme complexe
			usuelle au carré est un anneau euclidien.
	\end{enumerate}
\end{exemple}

La proposition suivante est très importante pour définir les idéaux d'un anneau
commutatif.

\begin{proposition}
	Tout anneau euclidien est principal, c'est-à-dire que tous ses idéaux sont
	engendré par un élément.
\end{proposition}

\begin{proof}
	Prenons un anneau euclidien A, ainsi qu'un idéal I de A. Si I = \GSset{0}
	ou \GSset{0, 1}, on a fini (\GSset{0} est un idéal engendré par 0, et
	\GSset{0, 1} = I car $1 \in I$.

	Sinon, I possède deux éléments x et y non nul. Nous pouvons supposer, sans
	perte de généralité\footnote{Si y ne vérifie pas la minimalité, on prend z
	tel que N(z) est minimal, et on remplace dans la preuve x par y et y par
z.}, que y est tel que N(y) = $\displaystyle min_{a \in I_{0}}$\GSset{N(a)} (en
effet, l'ensemble des N(a) est un sous ensemble de N, donc il possède un
minimum).
	On a, comme A est euclidien, qu'il existe q et r appartenant à A tel que $x
	= qy + r$, avec N(r) < N(y) ou r = 0.
	On a alors $r = x - qy$ avec $x \in I$, $qy \in I$ (comme $q \in A$ et $y
	\in I$ et I idéal), donc $r \in I$ comme I idéal.
	
	Or, si $N(r) < N(y)$ avec r non nul, cela contredirait la minimalité de N(y)
	car $r \in I_{0}$. Donc r est nul.

	On a alors que $x = qy$, c'est-à-dire que si on prend à chaque fois deux
	éléments quelconques de I, l'un est multiple de l'autre, c'est-à-dire que
	I est principal.
\end{proof}

\begin{corollary}
	Les idéaux de $\integer$ (resp $\integer$[i]) sont les n$\integer$ où $n \in
	\integer$ (resp $(a+bi) \integer$[i] où $a, b \in \integer$).
\end{corollary}
\end{document}


